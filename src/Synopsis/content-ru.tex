%!TEX root = ../dissertation.tex

\section*{Общая характеристика работы} %\addcontentsline{toc}{section}{Общая характеристика работы}

%!TEX root = ../dissertation.tex
\textbf{Актуальность темы.} 

\textbf{Степень разработанности темы.}

\textbf{Целью} исследования является разработка точных методов генерации детерминированных конечных автоматов по имеющимся примерам поведения с использованием сокращения пространства поиска при решении задачи выполнимости.

Для~достижения указанной цели определены следующие \textbf{задачи}:
\begin{enumerate}
  \item Разработать точный метод построения ДКА по имеющимся примерам поведения с использованием <<компактных>> предикатов нарушения симметрии на основе алгоритма обхода графа в ширину.
  \item Разработать методы сокращения пространства поиска при решении задачи выполнимости для 
  построения ДКА по имеющимся примерам поведения, основанные на структурных особенностях искомого автомата.
  \item Разработать точный метод построения всех возможных ДКА по имеющимся примерам поведения с использованием программных средств решения задачи выполнимости.
  \item Разработать комбинированный точный метод построения ДКА по имеющимся примерам поведения на основе сведения к задаче выполнимости и с использованием подхода уточнения абстракции по контрпримерам.
  \item Разработать программный комплекс, реализующий все вышеуказанные методы.
  \item Провести экспериментальные исследования для оценки применимости и времени работы разработанных и реализованных методов, изложенных выше. 
\end{enumerate}

\textbf{Объект исследования}~{---} методы генерации конечных автоматных моделей.

\textbf{Предмет исследования}~{---} точные методы генерации детерминированных конечных автоматов по имеющимся примерам поведения.

\textbf{Соответствие паспорту специальности.} Данная диссертация соответствует пункту 5 <<\textbf{Разработка и исследование моделей и алгоритмов анализа данных, обнаружения закономерностей в данных и их извлечениях разработка и исследование методов и алгоритмов анализа текста}, устной речи и изображений.>>.

\emph{??? пункт 10 <<\textbf{Разработка основ} математической теории языков и грамматик, \textbf{теории конечных автоматов} и теории графов.>>}

\emph{??? пункт 7 <<\textbf{Разработка методов} распознавания образов, фильтрации, \textbf{распознавания и синтеза} изображений, \textbf{решающих правил}. Моделирование формирования эмпирического знания.>>}

\emph{??? пункт 3 <<Исследование методов и разработка средств кодирования информации в виде данных. Принципы создания языков описания данных, языков манипулирования данными, языков запросов. \textbf{Разработка и исследование моделей данных и новых принципов их проектирования}>>.}

\textbf{Основные положения, выносимые на~защиту:}
\begin{enumerate}
  \item Разработан точный метод построения ДКА по имеющимся примерам поведения с использованием <<компактных>> предикатов нарушения симметрии на основе алгоритма обхода графа в ширину.
  \item Разработаны методы сокращения пространства поиска при решении задачи выполнимости для построения ДКА по имеющимся примерам поведения, основанные на структурных особенностях искомого автомата.
  \item Разработан точный метод построения всех возможных ДКА по имеющимся примерам поведения с использованием программных средств решения задачи выполнимости.
  \item Разработан комбинированный точный метод построения ДКА по имеющимся примерам поведения на основе сведения к задаче выполнимости и с использованием подхода уточнения абстракции по контрпримерам.
\end{enumerate}

\textbf{Научная новизна} заключается в~следующем:
\begin{enumerate}
  \item Впервые точный метод построения ДКА по имеющимся примерам поведения с использованием предикатов нарушения симметрии на основе алгоритма обхода графа в ширину был предложен научным руководителем В.И.~Ульянцевым при участии автора диссертации.
  Новизна предложенного в данной работе метода заключается в использовании <<компактных>> предикатов нарушения симметрии --- пропозициональное кодирование данных предикатов на языке задачи выполнимости содержит квадратичное относительно размера автомата количество дизъюнктов вместо кубического из оригинального метода.
  Разработанный метод помимо уменьшения размера булевой формулы обеспечивает статистически значимое преимущество скорости работы по сравнению с существующими точными методами.
  \item Впервые были предложены методы сокращения пространства поиска при решении задачи выполнимости для построения ДКА по имеющимся примерам поведения, основанные на структурных особенностях искомого автомата.
  Использование данных методов обеспечивает статистически значимое преимущество скорости работы по сравнению с существующими точными методами. 
  \item Впервые был предложен точный метод построения всех возможных ДКА по имеющимся примерам поведения с использованием программных средств решения задачи выполнимости.
  Новизна предложенного метода заключается в использовании предикатов нарушения симметрии на основе алгоритма обхода графа в ширину, что позволяет оставить для рассмотрения единственного представителя для каждого класса эквивалентности по изоморфизму.
  \item Впервые был предложен комбинированный точный метод построения ДКА по имеющимся примерам поведения на основе сведения к задаче выполнимости и с использованием подхода уточнения абстракции по контрпримерам.
  Новизна предложенного метода заключается в использовании только части имеющихся примеров поведения при построении ДКА, что обеспечивает статистически значимое преимущество скорости работы по сравнению с существующими точными методами в случаях, когда число имеющихся примеров поведения излишне велико относительно размера искомого автомата.
\end{enumerate}

\textbf{Методология и методы исследования.} В работе используются методы теории автоматов, теории сложности, дискретной математики, объектно-ориентированное программирование, методы проведения и анализа экспериментальных исследований

\textbf{Достоверность} полученных результатов, подтверждается корректным обоснованием постановок задач, точной формулировкой критериев, а также результатами проведенных экспериментальных исследований по использованию предложенных в диссертации методов.

\textbf{Теоретическая значимость работы} состоит в том, что для задачи генерации ДКА по имеющимся примерам поведения было предложено пропозициональное кодирование предикатов нарушения симметрии на основе алгоритма обхода графа в ширину, содержащее асимптотически меньшее количество дизъюнктов.
Также для задачи генерации ДКА по имеющимся примерам поведения были предложены методы по сокращению пространства поиска при решении задачи выполнимости.
Помимо этого, был предложен метод для решения задачи построения всех различных неизоморфных ДКА по имеющимся примерам поведения, для решения которой ранее эффективных методов не существовало.
Также был предложен метод, объединяющий в себе два различных подхода: сведение к задаче выполнимости и уточнение абстракции по контрпримерам.

\textbf{Практическая значимость работы} состоит в том, что разработанные методы и реализованные в рамках программного комплекса на основе данных методов алгоритмы  позволяют ускорить и улучшить генерацию ДКА по имеющимся примерам поведения, что позволяет решать задачи более высокой сложности, выраженной в размере искомого автомата, в размере алфавита, в недостаточном или избыточном покрытии автомата примерами поведения.

\textbf{Участие в научно-исследовательских работах.} 
Результаты диссертации использовались при проведении НИР <<Разработка эффективных методов машинного обучения для построения детерминированных конечных автоматов на основе решения задачи выполнимости>> (грант РФФИ <<Мой первый грант>> 18-37-00425, 2018--2020 гг.) под руководством автора диссертации.

Полученные результаты также использовались при проведении следующих НИР:
\begin{itemize}
  \item <<Биоинформатика, машинное обучение, технологии программирования, теория кодирования, проактивные системы>> (субсидия \mbox{074-U01} в рамках государственной финансовой поддержки ведущих университетов Российской Федерации, 2013--2017 гг.);
  \item <<Методы, модели и технологии искусственного интеллекта в биоинформатике, социальных медиа, киберфизических, биометрических и речевых системах>>, (субсидия \mbox{08-08} в рамках государственной финансовой поддержки ведущих университетов Российской Федерации, 2018--2020 гг.).
\end{itemize}

\textbf{Апробация результатов работы.}
Основные результаты работы докладывались на следующих конференциях и семинарах:
\begin{enumerate}
  \item 9\textsuperscript{th} International Conference on Language and Automata Theory and Applications (LATA 2015). 2015, Ницца, Франция.
  \item 6\textsuperscript{th} International Symposium ``From Data to Models and Back (DataMod)''. 2017, Тренто, Италия.
  \item 13\textsuperscript{th} International Conference on Language and Automata Theory and Applications (LATA 2019). 2015, Санкт-Петербург.
  \item IV-VII Всероссийский Конгресс молодых ученых. 2015-2018, Санкт-Петербург.
  \item IX Конгресс молодых ученых. 2020, Санкт-Петербург.
  \item XLVI Научная и учебно-методическая Конференция Университета \mbox{ИТМО}. 2017, Санкт-Петербург.
  \item XLVIII Научная и учебно-методическая Конференция Университета ИТМО. 2019, Санкт-Петербург.
\end{enumerate}

\textbf{Личный вклад автора.}.
  Идея точного метода построения ДКА по имеющимся примерам поведения с использованием <<компактных>> предикатов нарушения симметрии на основе алгоритма обхода графа в ширину принадлежит совместно автору диссертации и Ж.~Маркешу-Сильве, реализация алгоритма на базе предложенного метода принадлежит лично автору диссертации, проведение вычислительных экспериментов принадлежит совместно автору диссертации и А.И.~Игнатьеву.
  Идея методов сокращения пространства поиска при решении задачи выполнимости для построения ДКА по имеющимся примерам поведения, основанные на структурных особенностях искомого автомата принадлежит совместно автору диссертации и Ж.~Маркешу-Сильве, реализация алгоритмов на базе предложенных методов принадлежит лично автору диссертации, проведение вычислительных экспериментов принадлежит совместно автору диссертации и А.И.~Игнатьеву.
  Идея точного метода построения всех возможных ДКА по имеющимся примерам поведения с использованием программных средств решения задачи выполнимости принадлежит совместно автору диссертации и научному руководителю В.И.~Ульянцеву, реализация алгоритма на базе предложенного метода и проведение вычислительных экспериментов принадлежит лично автору.
  Идея комбинированного точного метода построения ДКА по имеющимся примерам поведения на основе сведения к задаче выполнимости и с использованием подхода уточнения абстракции по контрпримерам принадлежит совместно автору диссертации и научному руководителю В.И.~Ульянцеву, реализация алгоритма на базе предложенного метода и проведение вычислительных экспериментов принадлежит лично автору.

\textbf{Публикации.}
Основные результаты по теме диссертации изложены в девяти публикациях, из них три опубликованы в изданиях, индексируемых в базе цитирования Scopus, одна публикация издана в журнале, рекомендованном ВАК.
Также у автора диссертации имеются две публикации по другим темам, из которых одна связана с машинным обучением, другая с построением автоматных моделей для кибер-физических систем, обе опубликованы в изданиях, индексируемых в базе цитирования Scopus.

\section*{Содержание работы}

Во \textbf{введении} диссертационной работы обоснована актуальность проводимых исследований. Сформулированы цель, задачи и положения, выносимые на защиту. Изложена научная новизна и практическая значимость результатов, полученных в диссертационной работе.

%------

\textbf{\underline{Первая глава}} диссертации посвящена обзору предметной области и результатов существующих исследований, посвященных генерации детерминированных конечных автоматов.
Кроме того, в первой главе приведены терминология, основные определения и известные результаты ряда разделов информатики, необходимых для описания предлагаемых в диссертации методов и алгоритмов.

\insection{\ref{sec:review:sat}} приведена формальная постановка задачи выполнимости булевой формулы , даны необходимые определения, и представлено краткое описание основных подходов к ее решению.
Также в данном разделе описан подход к решению NP-трудных задач с помощью полиномиального сведения таких задач к задаче выполнимости.
Помимо этого, приводится краткий обзор существующих программных средств решения SAT.

Булева формула задана в \emph{конъюнктивно-нормальной форме} (КНФ), если является конъюнкцией дизъюнктов~--- множества литералов, связанных дизъюнкцией.
\emph{Задача выполнимости булевой формулы} (\emph{задача выполнимости}, \emph{boolean satisfiability problem}~{---} SAT) заключается в определении, существует ли выполняющая подстановка для некоторой булевой формулы заданной в КНФ.
Задача выполнимости является исторически первой, для которой была доказана NP-трудность~--- любую задачу из класса NP можно за полиномиальное время свести к SAT.
Данный факт объясняет актуальность разработки все более эффективных программных средств для решения задачи выполнимости.
Ежегодно в научном сообществе проходят соревнования по выяснению лучшего программного средства для решения SAT, что также способствует постоянному развитию данной области.
В основе современных программных средств лежит стратегия \emph{управляемого конфликтами обучения дизъюнктов} (\emph{conflict-driven clause learning}, CDCL).

Подход к решению задачи из класса NP, когда разрабатывается сведение к SAT и затем используется современное программное средство для поиска выполняющей подстановки, зачастую оказывается заметно эффективнее и проще, чем разработка применимого на практике метода, решающего непосредственно исходную задачу.
Еще одним немаловажным достоинством такого подхода является тот факт, что достаточно единожды написать сведение к SAT, а затем без прикладывания каких-либо усилий пользоваться развитием программных средств, выбирая самое эффективное из них.

\insection{\ref{sec:review:cegar}} приведено описание алгоритма уточнения абстракции по контрпримерам (counterexample guided abstraction refinement~{---} CEGAR). 
Методы, использующие данный алгоритм применимы в ситуации, когда необходимо построить модель, соответствующую заданным требованиям, имея при этом доступ к некоторой проверяющей системе~--- оракулу. 
На начальном шаге строится некоторая, возможно, случайная модель.
Затем начинается итеративный процесс уточнения имеющейся модели~--- на каждом шаге текущая модель проверяется оракулом на соответствие заданным требованиям. 
Если проверка проходит успешно, то модель найдена.
Иначе, оракул сообщает один или несколько контрпримеров, которые затем используются для уточнения модели.
\inote{картинку/схему/псевдокод}

\insection{\ref{sec:review:dfa-def}} приведены определение детерминированного конечного автомата (ДКА) и необходимые в диссертационной работе определения и свойства, связанные с ним.
\inote{картинка с ДКА}
\needtodo{данный раздел надо объединить со следующим}.

\insection{\ref{sec:review:dfa-inf}} приведена постановка задачи генерации детерминированных конечных автоматов по примерам поведения. 
Примерами поведения некоторого ДКА $\mathcal{D}$ называется множество слов, состоящих из символов алфавита автомата, про каждое из которых известно, принадлежит ли оно языку автомата $\mathcal{L}\left(\mathcal{D}\right)$ или нет.
Задача генерации ДКА по примерам поведения заключается в поиске автомата минимального размера (с минимальным числом состояний), соответствующего имеющимся примерам поведения.
Ранее было доказано, что настоящая задача является NP-полной, ровно как и задача генерации ДКА любого фиксированного размера, соответствующего имеющимся примерам поведения.
\inote{добавить про бритву Оккама, чтобы обосновать минимальность?}
\inote{пример/картинка вида ``данные $\rightarrow$ ДКА''}

\insection{\ref{sec:review:heuristic-dfa-inf}} приведен обзор существующих эвристических и метаэвристических методов генерации детерминированных конечных автоматов по примерам поведения. 

\inote{В основе эвристических алгоритмов лежит алгоритм \emph{слияния состояний} (\emph{state merging}), заключающийся в том, что на каждом шаге алгоритма выбирается несколько состояний расширенного префиксного дерева \inote{а про него не было ни слова, может надо сказать до этого?}, которые объединяются в одно и рекурсивно устраняется возникающая недетерминированность. 
Одним из самых эффективных представителей данной группы является метод \emph{объединения состояний на основе свидетельств} (\emph{evidence-driven state merging}, EDSM).}
Среди эвристических алгоритмов можно выделить алгоритм \emph{объединения состояний на основе свидетельств} (\emph{evidence-driven state merging}~--- EDSM).
Среди метаэвристических алгоритмов можно выделить эволюционные стратегии, генетические алгоритмы и муравьиные алгоритмы.
\inote{возможно пару слов о них и точно про их недостатки}
Следует отметить, что данные походы являются неточными~--- ими не гарантируется, что найденный автомат содержит минимальное возможное число состояний, а иногда вообще не гарантируется, что какой-то автомат будет найден.
\needtodo{добавить про основные работы/ученых} 


\insection{\ref{sec:review:sat-dfa-inf}} приведен обзор существующих методов генерации детерминированных конечных автоматов по примерам поведения, основанных на сведении к SAT. В отличие от эвристических и метаэвристических подходов, данные методы являются точным~--- гарантируется, что автомат, соответствующий примерам поведения, будет построен за конечное время и будет содержать минимальное возможное число состояний.

Первым шагом рассматриваемых методов является построение расширенного префиксного дерева (augmented prefix tree acceptor~--- APTA)~--- древовидной структуры данных, основанной на обычном префиксном дереве, в которой каждая вершина либо не помечена, либо помечена как допускающая или отвергающая.
\inote{рисунок АПТЫ}

Далее, начиная с некоторой нижней оценки на размер~--- в простейшем случае с единицы~--- происходит поиск автомата текущего размера, соответствующего построенному расширенному префиксному дереву. 
Данный процесс продолжается до тех пор, пока не будет найден ДКА, удовлетворяющий заданным требованием.
Итеративный перебор размера от меньшего к большему гарантирует, что найденный автомат имеет минимальный размер.
Как уже было сказано, задача поиска ДКА конкретного размера по заданным словарям принадлежит классу NP, а значит, может быть решена путем сведения к некоторой NP-трудной задаче.
Самым производительным точным методом до недавнего времени являлся DFASAT~\cite{heule-icgi10}, в котором авторы предложили сначала свести задачу генерации ДКА к задаче раскраски графа~--- необходимо раскрасить расширенное префиксное дерево в минимальное количество цветов так, чтобы все вершины одного цвета объединялись в одно состояние,~--- которую затем свести к задаче выполнимости.
Развитием данного метода занимался научный руководитель диссертанта В.~И.~Ульянцев в своей диссертационной работе, где было предложено использование предикатов нарушения симметрии на основе алгоритма обхода графа в ширину.\inote{ссылку}
Дальнейшему улучшению метода посвящена настоящая диссертация.

\needtodo{схема или псевдокод метода}

\insection{\ref{sec:review:sym-breaking}} приведен обзор существующих подходов к сокращению пространства поиска при генерации детерминированных конечных автоматов по примерам поведения.


\inote{возможно кратко про ИГ и клику; точно про изоморфизм, симметрию и БФС}

\insection{\ref{sec:review:tasks}} на основе результатов обзора формулируются задачи, решаемые в настоящей диссертационной работе.

%------

Во \textbf{\underline{второй главе}} настоящей диссертации описываются разработка, реализация и экспериментальные исследования методов генерации детерминированных конечных автоматов с использованием различных подходов к сокращению пространства поиска при решении задачи выполнимости.

\insection{\ref{sec:space:dfs}} приведено описание разработанных предикатов нарушения симметрии на основе алгоритма обхода графа в глубину (\emph{depth-first search}~--- DFS). 
Использование предикатов нарушения симметрии, задающих BFS нумерацию автомата, позволило значительно улучшить производительность метода DFASAT.
Логичной следующей задачей научного исследования было разработать предикаты нарушения симметрии на основе алгоритма DFS и метод, использующий их.
\inote{Нужны ли примеры DFS предикатов?}

Разработанные предикаты выражаются КНФ-формулой, содержащей $\mathcal{O}\left(M^{4} + M^{3} \times L^{2}\right)$ факториалов, что на фактор $M$ больше чем предикаты, задающие BFS нумерацию ДКА.

\insection{\ref{sec:space:tight}} приведено описание разработанных компактных предикатов нарушения симметрии на основе алгоритма BFS. \inote{Надо же как-то сказать, что не только предикаты, но и метод с их использованием разработан?} \inote{Рассказать, что значит ``компактность'' здесь. Привести пример того, как удалось сделать их компактнее}

\insection{\ref{sec:space:pruning}} приведено описание разработанных подходов к сокращению пространства поиска при генерации детерминированных конечных автоматов, основанных на особенностях структуры BFS дерева, а также на связях между расширенным префиксным деревом и искомом ДКА. \inote{Рисунок полного BFS дерева, пример особенностей его нумерации + рисунок и описание связи между APTA и DFA}

\insection{\ref{sec:space:results}} приведены описание разработанного  на языке \emph{python} программного средства \texttt{DFA-Inductor-py}, предназначенного для генерации детерминированных конечных автоматов, описание реализации разработанных методов как частей данного программного средства, а также результаты экспериментальных исследований всех разработанных методов. \inote{Кратко про средство} \inote{кратко описать результаты (dfs --- говно, а остальное --- ОК + таблицы и графики}.

%------

В \textbf{\underline{третьей главе}} настоящей диссертации описываются разработка, реализация и экспериментальные исследования комбинированного метода генерации детерминированных конечных автоматов на основе сведения к задаче выполнимости и с использованием подхода уточнения абстракции по контрпримерам.

\insection{\ref{sec:cegar:motivation}} приведено исследование границ применимости предложенных в предыдущих главах методов в зависимости от размера расширенного префиксного дерева. \inote{рассказать про то, что при большой APTA, формула тоже большая, солвер не тянет, а вот если взять часть примеров, то все ок}

\insection{\ref{sec:cegar:cegar-algo}} приведено описание разработанного комбинированного метода генерации ДКА по примерам поведения на основе сведения к SAT и с использование CEGAR. \inote{кратко описать метод + схема/псевдокод}.

\insection{\ref{sec:cegar:results}} приведены описание реализации разработанного метода как части программного средства \texttt{DFA-Inductor-py} и результаты экспериментальных исследований разработанного метода. \inote{рассказать про результаты --- бэктрекинг - не оч, а сат -- оч!}

%------

В \textbf{\underline{четвертой главе}} настоящей диссертации дается постановка задачи генерации всех различных детерминированных конечных автоматов по заданным примерам поведения, а затем описываются разработка, реализация и экспериментальные исследования двух методов, решающих поставленную задачу. 

\insection{\ref{sec:findall:problem}} приведена формальная постановка задачи генерации всех различных ДКА по примерам поведения. \inote{рассказать еще раз, что такое изоморфные автоматы, и почему их нельзя сгенерировать без идеальных предикатов.} \inote{привести саму постановку задачи}

\insection{\ref{sec:findall:SAT-based}} приведено описание метода генерации всех неизоморфных ДКА, основанный на сведении к SAT. \inote{кратко рассказать про метод, блокирующий дизъюнкт, инкрементальный солвер} \inote{схема/псевдокод}

\insection{\ref{sec:findall:backtracking}} приведено описание переборного метода генерации всех неизоморфных ДКА. \inote{рассказать, зачем этот метод нужен} \inote{схема/псевдокод}

\insection{\ref{sec:findall:results}} приведены описание реализации разработанных методов как частей программного средства \texttt{DFA-Inductor-py} и результаты экспериментальных исследований разработанного метода. \inote{рассказать про результаты, таблички и вот это все}

%------

%------

В \textbf{заключении} приведены основные результаты работы, которые заключаются в следующем:

%!TEX root = ../dissertation.tex
\begin{enumerate}
  \item .
  \item .
  
\end{enumerate}

...