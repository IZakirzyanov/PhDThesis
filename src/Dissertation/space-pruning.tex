%!TEX root = ../dissertation.tex

\chapter{Подходы к сокращению пространства поиска, основанные на структурных особенностях автомата} 
\label{sec:pruning}

В данной главе предлагаются новые методы по сокращению пространства поиска в задаче генерации ДКА минимального размера по заданным словарям. 
Данные методы не являются необходимыми для нахождения соответствующего автомата, но помогают сделать это быстрее.
В основе предлагаемых методов лежит использование структурных особенностей BFS пронумерованного ДКА, а также связь между расширенным префиксным деревом и ДКА.

\section{Полное дерево обхода в ширину}
\label{sec:pruning:bfs-tree}

\inote{рисунок дерева}

На рисунке \inote{ссылка} показано полное BFS дерево, построенное по некоторому автомату.
Данное дерево является полным, так как у каждой его внутренней вершины имеется по $L$ детей.
Тогда данное дерево показывает максимально возможные номера, которые могут быть у детей некоторого состояния $d_{i}$.
Действительно, нельзя добавить в данное дерево новые вершины, которые будут иметь номер между $i$ и $i \cdot L + 1$, так как все возможные позиции заняты.
В то же время, если удалить какие-то из вершин правее или ниже вершины $d_{i}$, то номера детей могут только уменьшиться.
Далее будут представлены дополнительные ограничения, которые следуют из рисунка \inote{ссылка}.

\paragraph{Сокращение области определения родительских переменных.}
У некоторого состояния $d_{i}$, где $1 \leq i < M$, детьми в BFS дереве могут быть только состояния с номерами от $i + 1$ до $min(i \cdot L + 1, M)$.
Так как в BFS дереве номер ребенка всегда больше номера родителя, то нижняя граница тривиальна.
Рисунок \inote{ссылка} иллюстрирует обоснование верхней границы.
Действительно, можно доказать по индукции, что состояния на $k$-ом уровне имеют номера от $\sum_{r = 0}^{k - 1}L^{r} + 1$ до $\sum_{r = 0}^{k}L^{r}$.
База индукции при $k = 0$, очевидно, верна.
Если для некоторого слоя $k$ утверждение выше верно, то для слоя $k + 1$ верно, что нумерация состояний на нем начинается с $\sum_{r = 0}^{k - 1}L^{r} + 1$, а всего вершин $\left(\sum_{r = 0}^{k}L^{r} - \left(\sum_{r = 0}^{k - 1}L^{r} + 1\right) + 1\right) \cdot L = L^{k} * L = L^{k + 1}$, из чего следует, что последнее состояние имеет номер $\sum_{r = 0}^{k}L^{r} + L^{k + 1} = \sum_{r = 0}^{k + 1}L^{r}$.

\inote{возможно, оформить в виде полноценной теоремы и доказательства.}

Выразить данное свойство можно, либо определив переменные для соответствующей области определения~{---} $\{p_{j,i}\}_{1 \leq i < j \leq min(i \cdot L + 1, M)}$, либо в явном виде указав, что $p_{j,i} = 0$ при $j > i \cdot L + 1$.

\paragraph{Сокращение области определения переменных переходов и переменных наличия переходов.}
Помимо закономерностей между номерами родителей и детей в BFS пронумерованном автомате, можно заметить более общую закономерность относительно переходов.
Из состояния $d_{i}$ в BFS пронумерованном автомате не может в принципе существовать перехода в состояние $d_{j}$ если $j > i \cdot L + 1$.
Действительно, из доказанного в предыдущей секции следует, что у состония $d_{j}$ родителем должно быть состояние $d_{k}$, где $k > i$.
Но, если существует из состояния $d_{i}$ существует переход в состояние $d_{j}$, то по принципу BFS обхода родителем состояния $d_{j}$ должно быть состояние $d_{k}$, где $k \leq i$.
Получившееся противоречие доказывает исходное утверждение.
Таким образом, можно сделать заключение, что $y_{i,l,j} = 0$ при $j > i \cdot L + 1; l \in \Sigma$.

Как следствие, для переменных наличия переходов верно, что $t_{i, j} = 0$ при $j > i \cdot L + 1$.

