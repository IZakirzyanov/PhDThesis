%!TEX root = ../dissertation.tex

\chapter{Подходы к сокращению пространства поиска, основанные на структурных особенностях автомата} 
\label{sec:pruning}

В данной главе предлагаются новые методы по сокращению пространства поиска в задаче генерации ДКА минимального размера по заданным словарям. 
Данные методы не являются необходимыми для нахождения соответствующего автомата, но помогают сделать это быстрее.
В основе предлагаемых методов лежит использование структурных особенностей BFS-пронумерованного ДКА, а также связь между расширенным префиксным деревом и ДКА.

\section{Полное дерево обхода в ширину}
\label{sec:pruning:bfs-tree}

\inote{рисунок дерева}

На рисунке \inote{ссылка} показано полное BFS дерево, построенное по некоторому автомату.
Данное дерево является полным, так как у каждой его внутренней вершины имеется по $L$ детей.
Тогда данное дерево показывает максимально возможные номера, которые могут быть у детей некоторого состояния $r$.
Действительно, нельзя добавить в данное дерево новые вершины, которые будут иметь номер между $r$ и $rL + 1$, так как все возможные позиции заняты.
В то же время, если удалить какие-то из вершин правее или ниже вершины $r$, то номера детей могут только уменьшиться.
Далее будут представлены дополнительные ограничения, которые следуют из рисунка \inote{ссылка}.

\paragraph{Новые области определения переменных $p_{j,i}$.}

