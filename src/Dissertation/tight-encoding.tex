%!TEX root = ../dissertation.tex

\chapter{Компактные предикаты нарушения симметрии} 
\label{sec:tight}

Кодирование свойства BFS-пронумерованности автомата, описанное в разделе~\ref{sec:review:sym-breaking:bfs-based}, состоящее из $\mathcal{O}\left(M^{3} + M^{2} \times L^{2}\right)$ дизъюнктов, слабо применимо на практике для ДКА большого размера, то есть при большом $M$. 
В данной главе описывается как модифицировать предикаты нарушения симметрии, рассмотренные ранее, таким образом, что для их булевого кодирования понадобится только $\mathcal{O}\left(M^{2} \times L\right)$ дизъюнктов.

\inote{подумать о подразделах}

%----------------------------------------------------------------------------------------

\section{Ревизия существующего кодирования}
\label{sec:tight:review}

Повторно кратко приведем наборы дизъюнктов, с помощью которых кодируются предикатов нарушения симметрии на основе алгоритма BFS, приведенные в разделе~\ref{sec:review:sym-breaking:bfs-based}.

\begin{equation}
\label{eq:t-def}
  \bigwedge_{1 \leq i < j \leq M} \left(t_{i,j} \leftrightarrow y_{i,l_{1},j} \vee y_{i,l_{2},j} \vee \ldots \vee y_{i,l_{L},j} \right)
\end{equation}
%
\begin{equation}
\label{eq:p-def}
  \bigwedge_{1 \leq i < j \leq M} \left(p_{j,i} \leftrightarrow t_{i,j} \wedge \neg t_{i - 1,j} \wedge \neg t_{i - 2, j} \wedge \ldots \wedge \neg t_{1,j}\right)
\end{equation}
%
\begin{equation}
\label{eq:p-alo}
  \bigwedge_{2 \leq j \leq M} \left(p_{j,1} \vee p_{j,2} \vee \ldots \vee p_{j,j - 1}\right)
\end{equation}
%
\begin{equation}
\label{eq:p-order}
  \bigwedge_{1 \leq k < i < j \leq M} \left(p_{j,i} \rightarrow \neg p_{j + 1, k}\right)
\end{equation}
%
\begin{equation}
\label{eq:m-def}
  \bigwedge_{1 \leq i < j \leq M} \bigwedge_{1 \leq n \leq L} \left(m_{i,l_{n},j} \leftrightarrow y_{i,l_{n},j} \wedge \neg y_{i,l_{n - 1}, j} \wedge \neg y_{i,l_{n - 2}, j} \wedge \ldots \wedge \neg y_{i,l_{1},j} \right)
\end{equation}
%
\begin{equation}
\label{eq:m-order}
  \bigwedge_{1 \leq i < j \leq M} \bigwedge_{1 \leq k < n \leq L} \left(p_{j,i} \wedge p_{j + 1, i} \wedge m_{i,l_{n}, j} \rightarrow \neg m_{i, l_{k}, j + 1}\right)
\end{equation}

Изучив формулы, приведенные выше, можно заключить, что:
\begin{enumerate}
  \item формула~\eqref{eq:t-def} содержит $\mathcal{O}\left(M^{2} \times L\right)$ дизъюнктов;
  \item формула~\eqref{eq:p-def} содержит $\mathcal{O}\left(M^{3}\right)$ дизъюнктов;
  \item формула~\eqref{eq:p-alo} содержит $\mathcal{O}\left(M\right)$ дизъюнктов;
  \item формула~\eqref{eq:p-order} содержит $\mathcal{O}\left(M^{3}\right)$ дизъюнктов;
  \item формула~\eqref{eq:m-def} содержит $\mathcal{O}\left(M^{2} \times L^{2}\right)$ дизъюнктов;
  \item формула~\eqref{eq:m-order} содержит $\mathcal{O}\left(M^{2} \times L^{2}\right)$ дизъюнктов.
\end{enumerate}
%
Таким образом, будет предложено новое кодирование для свойств, ранее выраженных формулами~\eqref{eq:p-def}, \eqref{eq:p-alo}, \eqref{eq:m-def} и \eqref{eq:m-order}.

Прежде всего, заметим, что формула~\eqref{eq:p-alo} задает свойство, что у каждого состояния $d_{j}$ (кроме начального) автомата $\mathcal{D}$ существует как минимум один родитель в дереве BFS, при этом с меньшим номером.
Однако, надо заметить, что по определению в любом дереве у любой вершины кроме корня существует ровно один родитель.
Тогда, можно добавить ограничение, задающее свойство, что у каждого состояния $d_{j}$ (кроме начального) автомата $\mathcal{D}$ существует не более одного родителя в дереве BFS, при этом с меньшим номером.
В совокупности два данных ограничения зададут вышеупомянутое свойство об единственности родителя.
Заметим, что ограничение, задающее свойство, что не более чем одна из $M$ переменных истинна, может быть выражено через $\mathcal{O}\left(M^2\right)$ или $\mathcal{O}\left(M\times \log M\right)$ дизъюнктов, что укладывается в целевой размер формулы $\mathcal{O}\left(M^{2} \times L\right)$ дизъюнктов.
\inote{либо написать как закодировать прямо тут, либо сослаться на раздел, где про это напишу}.

Фактически, новое ограничение можно записать следующим образом.
\begin{equation}
\label{eq:p-sum-eq-one}
  \bigwedge_{1 < j \leq M} \sum_{i=1}^{j-1}p_{j,i}=1
\end{equation}

%----------------------------------------------------------------------------------------

\section{Определение родительских переменных}
\label{sec:tight:p-def}

Формула $$\bigwedge_{1 \leq i < j \leq M} \left(p_{j,i} \leftrightarrow t_{i,j} \wedge \neg t_{i - 1,j} \wedge \neg t_{i - 2, j} \wedge \ldots \wedge \neg t_{1,j}\right)$$ при преобразовании в КНФ выражается через $\mathcal{O}\left(M^{3}\right)$ дизъюнктов, так как обе переменные $i$ и $j$ имеют область допустимых значений размера $M$, а также правая часть формулы имеет длину $O\left(M\right)$.
Так как переменные $i$ и $j$ независимы,то, чтобы сократить количество дизъюнктов, нужно сократить правую часть формулы.
Для этого предлагается ввести новые булевы переменные $\{\mathit{ft}_{i,j}\}_{0 \leq i < j \leq M}$.
Переменная $\mathit{ft}_{i,j}$ истинна тогда и только тогда, когда все переменные $t_{k,j}$, где $1 \leq k \leq i$, ложны (\textbf{f}alse $\boldsymbol{t}$ variables~{---} ложные переменные $t$).
Иными словами, $\mathit{ft}_{i,j} \leftrightarrow \neg t_{i,j} \wedge \neg t_{i - 1, j} \wedge \ldots \neg t_{1,j}$. 
Для $i = 0$ в явном виде определим, что $\mathit{ft}_{0,j} = 1$ для любого $j$.
Стоит, однако, заметить, что определение переменных $\mathit{ft}_{i,j}$, приведенное выше, требует также $\mathcal{O}\left(M^3\right)$ дизъюнктов, что не решает изначальную проблему.

При этом, можно заметить, что значение переменной $\mathit{ft}_{i,j}$ зависит от значения всех переменных $\mathit{ft}_{k,j}$, где $k < i$. Тогда, можно определить переменные $\mathit{ft}_{i,j}$ рекурсивно:

\begin{equation}
\label{eq:ft-def}
  \mathit{ft}_{i,j} \leftrightarrow 
    \begin{cases} 
      1                               & i = 0, 1 \leq j \leq M \\
      \mathit{ft}_{i-1,j} \wedge \neg t_{i,j}  & 1 \leq i < j \leq M
    \end{cases} 
\end{equation}
%
При преобразовании в КНФ формула~\eqref{eq:ft-def} будет состоять из $\mathcal{O}\left(M^{2}\right)$ дизъюнктов.

Используя новые переменные $\mathit{ft}_{i,j}$ формулу~\eqref{eq:p-def} можно переписать следующим образом.
%
\begin{equation}
\label{eq:p-def-tight}
  \bigwedge_{1 \leq i < j \leq M} \left(p_{j,i} \leftrightarrow t_{i,j} \wedge \mathit{ft}_{i-1,j}\right)
\end{equation}
%
В данной формуле решена проблема длинной правой части и общее число дизъюнктов, требуемых для кодирования переменных $p_{j,i}$, также $\mathcal{O}\left(M^{2}\right)$. 

%----------------------------------------------------------------------------------------

\section{Порядок детей с помощью родительных переменных}
\label{sec:tight:p-order}

\inote{БЛЯ! Мы по поводу этого раздела много выясняли, спорили с Жуаном, как правильно тут все устроено и т.д. Потому что выглядит все реально сложно. И только сейчас я понял, что это можно сделать ровно также как в разделе~\ref{sec:tight:m-order}}
\inote{Переписать этот раздел}
\inote{И, вероятно, всю главу тогда можно единообразно переписать}

Формула $$\bigwedge_{1 \leq k < i < j \leq M} \left(p_{j,i} \rightarrow \neg p_{j + 1, k}\right)$$ при преобразовании в КНФ выражается через $\mathcal{O}\left(M^{3}\right)$ дизъюнктов, так как все три переменные $i$, $j$ и $k$ имеют домен размера $M$.
Переменные $i$ и $j$ являются независимыми, поэтому сократить размер всей формулы можно только избавившись от переменной $k$.

Фактически, данная формула задает ограничение, что родитель состояния $d_{j + 1}$ автомата $\mathcal{D}$ имеет номер не меньший, чем номер родителя состояния $d_{j}$. 
Также можно заметить, что учитывая ограничение, заданное формулой~\eqref{eq:p-sum-eq-one}, двоичное число $\mathbf{p_{j}}=\overline{p_{j,1}p_{j,2}\ldots p_{j,j-1}}$ состоит из $j - 2$ нулей и одной единицы.
Тогда рассматриваемая формула говорит, что в числе $\mathbf{p_{j + 1}}$ единственная единица стоит не левее чем в векторе $\mathbf{p_{j}}$ (и наоборот, что в числе $\mathbf{p_{j}}$ единственная единица стоит не правее чем в векторе $\mathbf{p_{j + 1}}$). 
Для удобства сравнения данных чисел, рассматривается расширенное число $\tilde{\mathbf{p_{j}}} = \overline{p_{j,1}p_{j,2}\ldots p_{j,j-1}0}$.
В контексте имеющейся формулы расширенное число не отличается от обычного, так как в нем все еще содержится ровно одна единица на том же месте, что и раньше (считая, слева), но теперь двоичные числа $\tilde{\mathbf{p_{j}}}$ и $\mathbf{p_{j+1}}$ имеют одинаковое число цифр.
Тогда исходная формула фактически задает ограничение $\tilde{\mathbf{p_{j}}} \geq \mathbf{p_{j + 1}}$.
С помощью $\mathbf{p_{j}}^{\mathbf{i}}$ далее в данном разделе будет обозначаться двоичное число, являющееся суффиксом числа $\mathbf{p_{j}}$, начинающимся с $i$-ой цифры и заканчивая последней: $\mathbf{p_{j}}^{\mathbf{i}}=\overline{p_{j,i}p_{j,i+1}\ldots p_{j,j - 1}}$.
Аналогично, $\tilde{\mathbf{p_{j}}}^{\mathbf{i}}=\overline{p_{j,i}p_{j,i+1}\ldots p_{j,j}}$.

Для сравнения необходимо ввести новые булевы переменные $\{\mathit{geq}_{j,i}\}_{1 < j \leq M, 1 \leq i \leq j + 1}$.
Переменная $\{\mathit{geq}_{j,i}\}$ истинна тогда и только тогда, когда число $\tilde{\mathbf{p_{j}}}^{\mathbf{i}}$ больше или равно (\textbf{g}reater or \textbf{eq}ual) чем число $\mathbf{p_{j + 1}}^{\mathbf{i}}$, а значит и единица во втором числе находится не левее чем в первом. 
Определить переменные $\mathit{geq}_{j,i}$ можно рекурсивно следующим образом:
%
\begin{equation}
\label{eq:geq-def}
  \mathit{geq}_{j,i} \leftrightarrow 
    \begin{cases} 
      1                               & i = j + 1, 1 \leq j \leq M \\
      \mathit{geq}_{j,i + 1} \wedge \left(p_{j,i} \leftrightarrow p_{j + 1, i}\right) \vee p_{j,i} \wedge \neg p_{j + 1, i}  & 1 \leq i \leq j \leq M
    \end{cases} 
\end{equation}

Ограничение $\mathit{geq}_{j,j + 1} = 1$ задает исходное значение для рекурсивного определения переменных $\mathit{geq}_{j,i}$.
Далее, число $\tilde{\mathbf{p_{j}}}^{\mathbf{i}}$ больше или равно числа $\mathbf{p_{j + 1}}^{\mathbf{i}}$, то есть $\mathit{geq}_{j,i} = 1$, если $i$-ый бит чисел $\tilde{\mathbf{p_{j}}}$ и $\mathbf{p_{j + 1}}$ совпадает, а для суффиксов $\tilde{\mathbf{p_{j}}}^{\mathbf{i + 1}}$ и $\mathbf{p_{j + 1}}^{\mathbf{i + 1}}$ верно, что первый больше либо равен второго, или если $i$-ый бит числа $\tilde{\mathbf{p_{j}}}$ равен единице, а числа $\mathbf{p_{j + 1}}$~{---} нулю.
Последнее верно, так как оба числа содержат по одной единице и вне зависимости от того, где находится единица во втором числе, правее или левее, число $\tilde{\mathbf{p_{j}}}^{\mathbf{i}}$ строго больше числа $\mathbf{p_{j + 1}}^{\mathbf{i}}$.

Для упрощения записи и уменьшения размера дизъюнктов можно ввести еще одни вспомогательные переменные $\{\mathit{peq}_{j,i}\}_{1 \leq i \leq j \leq M}$.
Переменная $\mathit{peq}_{j,i}$ истинна тогда и только тогда, когда $p_{j,i} = p_{j + 1,i}$.
%
\begin{equation}
\label{eq:peq-def}
  \mathit{peq}_{j,i} \leftrightarrow \left(p_{j,i} \leftrightarrow p_{j + 1, i}\right) 
\end{equation}


Тогда формула~\eqref{eq:geq-def} примет окончательный вид:
%
\begin{equation}
\label{eq:geq-def2}
  \mathit{geq}_{j,i} \leftrightarrow 
    \begin{cases} 
      1                               & i = j + 1, 1 \leq j \leq M \\
      \mathit{geq}_{j,i + 1} \wedge \mathit{peq}_{j,i} \vee p_{j,i} \wedge \neg p_{j + 1, i}  & 1 \leq i \leq j \leq M
    \end{cases} 
\end{equation}
%
При преобразовании в КНФ формула~\eqref{eq:geq-def2} будет состоять из $\mathcal{O}\left(M^{2}\right)$ дизъюнктов.

Используя новые переменные $\mathit{geq}_{i,j}$ формулу~\eqref{eq:p-order} можно переписать следующим образом.
%
\begin{equation}
\label{eq:p-def-tight}
  \bigwedge_{1 < j \leq M} \mathit{geq}_{j,1}
\end{equation}
%
Действительно, $\mathit{geq}_{j,1} = 1 \Leftrightarrow \tilde{\mathbf{p_{j}}} = \tilde{\mathbf{p_{j}}}^\mathbf{1} \geq \mathbf{p_{j + 1}}^{\mathbf{1}} = \mathbf{p_{j + 1}}$.

Таким образом, формула~\eqref{eq:p-def-tight} выражается через $\mathcal{O}\left(M\right)$ дизъюнктов, а формулы~\eqref{eq:peq-def} и~\eqref{eq:geq-def2}~{---} через $\mathcal{O}\left(M^{2}\right)$ дизъюнктов.

%----------------------------------------------------------------------------------------

\section{Определение переменных минимального символа}
\label{sec:tight:m-def}

Формула $$\bigwedge_{1 \leq i < j \leq M} \bigwedge_{1 \leq n \leq L} \left(m_{i,l_{n},j} \leftrightarrow y_{i,l_{n},j} \wedge \neg y_{i,l_{n - 1}, j} \wedge \neg y_{i,l_{n - 2}, j} \wedge \ldots \wedge \neg y_{i,l_{1},j} \right)$$ при преобразовании в КНФ выражается через $\mathcal{O}\left(M^{2} \times L^{2}\right)$ дизъюнктов, так как обе переменные $i$ и $j$ имеют область допустимых значений размера $M$, переменная $n$~{---} размера $L$, а также правая часть формулы имеет длину $O\left(L\right)$.
Так как переменные $i$, $j$ и $n$ независимы,то, чтобы сократить количество дизъюнктов, нужно сократить правую часть формулы.
Можно заметить, что данная формула по своей структуре аналогична той, что рассматривалась в разделе~\ref{sec:tight:p-def}.
Тогда аналогично можно ввести новые булевы переменные $\{\mathit{fy}_{i,l_{n},j}\}_{0 \leq i < j \leq M,0 \leq n \leq M}$.
Переменная $\mathit{fy}_{i,l_{n},j}$ истинна тогда и только тогда, когда все переменные $y_{i,l_{k},j}$, где $1 \leq k \leq n$, ложны (\textbf{f}alse $\boldsymbol{y}$ variables~{---} ложные переменные $y$).
Иными словами, $\mathit{fy}_{i,l_{n},j} \leftrightarrow \neg y_{i,l_{n},j} \wedge \neg y_{i, l_{n - 1}, j} \wedge \ldots \neg y_{i,l_{1},j}$. 
Для $n = 0$ в явном виде определим, что $\mathit{fy}_{i,l_{0},j} = 1$ для любых $i,j$.
Далее, аналогично тому, как это было сделано в разделе~\ref{sec:tight:p-def}, определим переменные $\mathit{fy}_{i,l_{n},j}$ рекурсивно.

\begin{equation}
\label{eq:fy-def}
  \mathit{fy}_{i,l_{n},j} \leftrightarrow 
    \begin{cases} 
      1                               & 1 \leq i < j \leq M, n = 0 \\
      \mathit{fy}_{i,l_{n - 1},j} \wedge \neg y_{i,l_{n},j}  & 1 \leq i < j \leq M, 1 \leq n \leq L
    \end{cases} 
\end{equation}
%
При преобразовании в КНФ формула~\eqref{eq:fy-def} будет состоять из $\mathcal{O}\left(M^{2} \times L\right)$ дизъюнктов.

Используя новые переменные $\mathit{fy}_{i,l_{n},j}$ формулу~\eqref{eq:m-def} можно переписать следующим образом.
%
\begin{equation}
\label{eq:m-def-tight}
  \bigwedge_{1 \leq i < j \leq M} \bigwedge_{1 \leq n \leq L} \left(m_{i,l_{n},j} \leftrightarrow y_{i,l_{n},j} \wedge \mathit{fy}_{i,l_{n - 1},j} \right)
\end{equation}
%
В данной формуле решена проблема длинной правой части и общее число дизъюнктов, требуемых для кодирования переменных $m_{i,l_{n},j}$, также $\mathcal{O}\left(M^{2} \times L\right)$. 

%----------------------------------------------------------------------------------------

\section{Порядок детей одного родителя}
\label{sec:tight:m-order}

Формула $$\bigwedge_{1 \leq i < j \leq M} \bigwedge_{1 \leq k < n \leq L} \left(p_{j,i} \wedge p_{j + 1, i} \wedge m_{i,l_{n}, j} \rightarrow \neg m_{i, l_{k}, j + 1}\right)$$ при преобразовании в КНФ выражается также через $\mathcal{O}\left(M^{2} \times L^{2}\right)$ дизъюнктов, так как обе переменные $i$ и $j$ имеют область допустимых значений размера $M$, а обе переменные $n$ и $k$~{---} размера $L$.
Переменные $i$,$j$ и $n$ являются независимыми, поэтому сократить размер всей формулы можно только избавившись от переменной $k$.
Можно заметить, что данная формула по своей структуре аналогична той, что рассматривалась в разделе~\ref{sec:tight:p-order}.
Тогда аналогично можно ввести новые булевы переменные $\{\mathit{fm}_{i,l_{n},j}\}_{0 \leq i < j \leq M,0 \leq n \leq M}$.
Переменная $\mathit{fm}_{i,l_{n},j}$ истинна тогда и только тогда, когда все переменные $m_{i,l_{k},j}$, где $1 \leq k \leq n$, ложны (\textbf{f}alse $\boldsymbol{m}$ variables~{---} ложные переменные $m$).
Иными словами, $\mathit{fm}_{i,l_{n},j} \leftrightarrow \neg m_{i,l_{n},j} \wedge \neg m_{i, l_{n - 1}, j} \wedge \ldots \neg m_{i,l_{1},j}$. 
Для $n = 0$ в явном виде определим, что $\mathit{fm}_{i,l_{0},j} = 1$ для любых $i,j$.
Далее, аналогично тому, как это было сделано в разделе~\ref{sec:tight:p-order}, определим переменные $\mathit{fm}_{i,l_{n},j}$ рекурсивно.

\begin{equation}
\label{eq:fm-def}
  \mathit{fm}_{i,l_{n},j} \leftrightarrow 
    \begin{cases} 
      1                               & 1 \leq i < j \leq M, n = 0 \\
      \mathit{fm}_{i,l_{n - 1},j} \wedge \neg m_{i,l_{n},j}  & 1 \leq i < j \leq M, 1 \leq n \leq L
    \end{cases} 
\end{equation}
%
При преобразовании в КНФ формула~\eqref{eq:fm-def} будет состоять из $\mathcal{O}\left(M^{2} \times L\right)$ дизъюнктов.

Используя новые переменные $\mathit{fm}_{i,l_{n},j}$ формулу~\eqref{eq:m-order} можно переписать следующим образом.
%
\begin{equation}
\label{eq:m-order-tight}
  \bigwedge_{1 \leq i < j < M} \bigwedge_{1 \leq n \leq L} \left(p_{j,i} \wedge p_{j + 1, i} \wedge m_{i,l_{n}, j} \rightarrow \neg \mathit{fm}_{i, l_{n - 1}, j + 1}\right)
\end{equation}
%
Таким образом общее число дизъюнктов, требуемых для ограничения порядка детей одного состояния, равняется $\mathcal{O}\left(M^{2} \times L\right)$. 
