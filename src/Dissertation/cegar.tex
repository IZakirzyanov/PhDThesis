%!TEX root = ../dissertation.tex

\chapter{Генерация детерминированных конечных автоматов при помощи уточнения абстракции по контрпримерам} 
\label{sec:cegar}

В настоящей главе описывается комбинированный метод генерации детерминированных конечных автоматов, на основе сведения к задаче выполнимости булевых формул и с использованием подхода уточнения абстракции по контрпримерам.

\section{Границы применимости (масштабируемость) предложенных методов в зависимости от размера расширенного префиксного дерева} % (fold)
\label{sec:cegar:motivation}

\inote{Методы, разработанные ранее супер! Но размер SAT формулы зависит от $N$ и $M$. Если $N$ (размер APTA) большое, то формула неудобоварима для солвера. Зачастую в примерах поведения содержится избыточная информация, которая ненужна для генерации соответствующего автомата. Вот, посмотрите, на пример. Куча примеров поведения, а автомат небольшой. Солвер не справляется, однако, если взять половину примеров поведения, то автомат строится за пару секунд. Но как понять какую половину нужно взять? Поможет CEGAR!}

\section{Метод генерации детерминированных конечных автоматов на основе сведения к задаче выполнимости и с использованием подхода уточнения абстракции по контрпримерам}
\label{sec:cegar:cegar-algo}

%----------------------------------------------------------------------------------------

\section{Реализация и экспериментальные исседования разработанного метода}
\label{sec:cegar:results}

%----------------------------------------------------------------------------------------

\subsection{Реализация разработанного комбинированного метода генерации детерминированных конечных автоматов}
\label{sec:cegar:results:impl}

%----------------------------------------------------------------------------------------

\subsection{Экспериментальные исследования разработанного метода комбинированного метода генерации детерминированных конечных автоматов}
\label{sec:cegar:results:cegar}

%----------------------------------------------------------------------------------------

\chresults{\ref{sec:cegar}}
