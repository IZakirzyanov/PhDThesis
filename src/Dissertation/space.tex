%!TEX root = ../dissertation.tex

\chapter{Сокращение пространства поиска при генерации детерминированных конечных автоматов с использованием сведения к задаче выполнимости}
\label{sec:space}

%------------------------------------------------------------------------------------------------------

\section{Предикаты нарушения симметрии на основе аглоритма обхода графа в глубину}
\label{sec:space:dfs}

%------------------------------------------------------------------------------------------------------

\section{Компактные предикаты нарушения симметрии на основе алгоритма обхода графа в ширину} 
\label{sec:space:tight}

Кодирование свойства BFS пронумерованности автомата, описанное в разделе~\ref{sec:review:sym-breaking:bfs-based}, состоящее из $\mathcal{O}\left(M^{3} + M^{2} \times L^{2}\right)$ дизъюнктов, слабо применимо на практике для ДКА большого размера, то есть при большом $M$. 
В данной главе описывается как модифицировать предикаты нарушения симметрии, рассмотренные ранее, таким образом, что для их булевого кодирования понадобится только $\mathcal{O}\left(M^{2} \times L\right)$ дизъюнктов.

\inote{подумать о подразделах}

%----------------------------------------------------------------------------------------

\subsection{Ревизия существующего кодирования}
\label{sec:space:tight:review}

В данном подразделе повторно кратко приводятся наборы дизъюнктов, с помощью которых кодируются предикатов нарушения симметрии на основе алгоритма BFS, приведенные в разделе~\ref{sec:review:sym-breaking:bfs-based}.

\begin{equation}
\label{eq:t-def}
  \bigwedge_{1 \leq i < j \leq M} \left(t_{i,j} \leftrightarrow y_{i,l_{1},j} \vee y_{i,l_{2},j} \vee \ldots \vee y_{i,l_{L},j} \right)
\end{equation}
%
\begin{equation}
\label{eq:p-def}
  \bigwedge_{1 \leq i < j \leq M} \left(p_{j,i} \leftrightarrow t_{i,j} \wedge \neg t_{i - 1,j} \wedge \neg t_{i - 2, j} \wedge \ldots \wedge \neg t_{1,j}\right)
\end{equation}
%
\begin{equation}
\label{eq:p-alo}
  \bigwedge_{2 \leq j \leq M} \left(p_{j,1} \vee p_{j,2} \vee \ldots \vee p_{j,j - 1}\right)
\end{equation}
%
\begin{equation}
\label{eq:p-order}
  \bigwedge_{1 \leq k < i < j \leq M} \left(p_{j,i} \rightarrow \neg p_{j + 1, k}\right)
\end{equation}
%
\begin{equation}
\label{eq:m-def}
  \bigwedge_{1 \leq i < j \leq M} \bigwedge_{1 \leq n \leq L} \left(m_{i,l_{n},j} \leftrightarrow y_{i,l_{n},j} \wedge \neg y_{i,l_{n - 1}, j} \wedge \neg y_{i,l_{n - 2}, j} \wedge \ldots \wedge \neg y_{i,l_{1},j} \right)
\end{equation}
%
\begin{equation}
\label{eq:m-order}
  \bigwedge_{1 \leq i < j \leq M} \bigwedge_{1 \leq k < n \leq L} \left(p_{j,i} \wedge p_{j + 1, i} \wedge m_{i,l_{n}, j} \rightarrow \neg m_{i, l_{k}, j + 1}\right)
\end{equation}

Изучив формулы, приведенные выше, можно заключить, что:
\begin{enumerate}
  \item формула~\eqref{eq:t-def} содержит $\mathcal{O}\left(M^{2} \times L\right)$ дизъюнктов;
  \item формула~\eqref{eq:p-def} содержит $\mathcal{O}\left(M^{3}\right)$ дизъюнктов;
  \item формула~\eqref{eq:p-alo} содержит $\mathcal{O}\left(M\right)$ дизъюнктов;
  \item формула~\eqref{eq:p-order} содержит $\mathcal{O}\left(M^{3}\right)$ дизъюнктов;
  \item формула~\eqref{eq:m-def} содержит $\mathcal{O}\left(M^{2} \times L^{2}\right)$ дизъюнктов;
  \item формула~\eqref{eq:m-order} содержит $\mathcal{O}\left(M^{2} \times L^{2}\right)$ дизъюнктов.
\end{enumerate}
%
Таким образом, будет предложено новое кодирование для свойств, ранее выраженных формулами~\eqref{eq:p-def}, \eqref{eq:p-alo}, \eqref{eq:m-def} и \eqref{eq:m-order}.

Прежде всего, заметим, что формула~\eqref{eq:p-alo} задает свойство, что у каждого состояния $d_{j}$ (кроме начального) автомата $\mathcal{D}$ существует как минимум один родитель в дереве BFS, при этом с меньшим номером.
Однако, надо заметить, что по определению в любом дереве у любой вершины кроме корня существует ровно один родитель.
Тогда, можно добавить ограничение, задающее свойство, что у каждого состояния $d_{j}$ (кроме начального) автомата $\mathcal{D}$ существует не более одного родителя в дереве BFS, при этом с меньшим номером.
В совокупности два данных ограничения зададут вышеупомянутое свойство об единственности родителя.
Заметим, что ограничение, задающее свойство, что не более чем одна из $M$ переменных истинна, может быть выражено через $\mathcal{O}\left(M^2\right)$ или $\mathcal{O}\left(M\times \log M\right)$ дизъюнктов, что укладывается в целевой размер формулы $\mathcal{O}\left(M^{2} \times L\right)$ дизъюнктов.
\inote{либо написать как закодировать прямо тут, либо сослаться на раздел, где про это напишу}.

Фактически, новое ограничение можно записать следующим образом.
\begin{equation}
\label{eq:p-sum-eq-one}
  \bigwedge_{1 < j \leq M} \sum_{i=1}^{j-1}p_{j,i}=1
\end{equation}

%----------------------------------------------------------------------------------------

\subsection{Определение родительских переменных}
\label{sec:space:tight:p-def}

Формула $$\bigwedge_{1 \leq i < j \leq M} \left(p_{j,i} \leftrightarrow t_{i,j} \wedge \neg t_{i - 1,j} \wedge \neg t_{i - 2, j} \wedge \ldots \wedge \neg t_{1,j}\right)$$ при преобразовании в КНФ выражается через $\mathcal{O}\left(M^{3}\right)$ дизъюнктов, так как обе переменные $i$ и $j$ имеют область допустимых значений размера $M$, а также правая часть формулы имеет длину $O\left(M\right)$.
Так как переменные $i$ и $j$ независимы,то, чтобы сократить количество дизъюнктов, нужно сократить правую часть формулы.
Для этого предлагается ввести новые булевы переменные $\{\mathit{ft}_{i,j}\}_{0 \leq i < j \leq M}$.
Переменная $\mathit{ft}_{i,j}$ истинна тогда и только тогда, когда все переменные $t_{k,j}$, где $1 \leq k \leq i$, ложны (\textbf{f}alse $\boldsymbol{t}$ variables~{---} ложные переменные $t$).
Иными словами, $\mathit{ft}_{i,j} \leftrightarrow \neg t_{i,j} \wedge \neg t_{i - 1, j} \wedge \ldots \neg t_{1,j}$. 
Для $i = 0$ в явном виде определим, что $\mathit{ft}_{0,j} = 1$ для любого $j$.
Стоит, однако, заметить, что определение переменных $\mathit{ft}_{i,j}$, приведенное выше, требует также $\mathcal{O}\left(M^3\right)$ дизъюнктов, что не решает изначальную проблему.

При этом, можно заметить, что значение переменной $\mathit{ft}_{i,j}$ зависит от значения всех переменных $\mathit{ft}_{k,j}$, где $k < i$. Тогда, можно определить переменные $\mathit{ft}_{i,j}$ рекурсивно:

\begin{equation}
\label{eq:ft-def}
  \mathit{ft}_{i,j} \leftrightarrow 
    \begin{cases} 
      1                               & i = 0, 1 \leq j \leq M \\
      \mathit{ft}_{i-1,j} \wedge \neg t_{i,j}  & 1 \leq i < j \leq M
    \end{cases} 
\end{equation}
%
При преобразовании в КНФ формула~\eqref{eq:ft-def} будет состоять из $\mathcal{O}\left(M^{2}\right)$ дизъюнктов.

Используя новые переменные $\mathit{ft}_{i,j}$ формулу~\eqref{eq:p-def} можно переписать следующим образом.
%
\begin{equation}
\label{eq:p-def-tight}
  \bigwedge_{1 \leq i < j \leq M} \left(p_{j,i} \leftrightarrow t_{i,j} \wedge \mathit{ft}_{i-1,j}\right)
\end{equation}
%
В данной формуле решена проблема длинной правой части и общее число дизъюнктов, требуемых для кодирования переменных $p_{j,i}$, также $\mathcal{O}\left(M^{2}\right)$. 

%----------------------------------------------------------------------------------------

\subsection{Порядок детей с помощью родительских переменных}
\label{sec:space:tight:p-order}

\inote{БЛЯ! Мы по поводу этого раздела много выясняли, спорили с Жуаном, как правильно тут все устроено и т.д. Потому что выглядит все реально сложно. И только сейчас я понял, что это можно сделать ровно также как в разделе~\ref{sec:space:tight:m-order}}
\inote{Переписать этот раздел}
\inote{И, вероятно, всю главу тогда можно единообразно переписать}

Формула $$\bigwedge_{1 \leq k < i < j \leq M} \left(p_{j,i} \rightarrow \neg p_{j + 1, k}\right)$$ при преобразовании в КНФ выражается через $\mathcal{O}\left(M^{3}\right)$ дизъюнктов, так как все три переменные $i$, $j$ и $k$ имеют домен размера $M$.
Переменные $i$ и $j$ являются независимыми, поэтому сократить размер всей формулы можно только избавившись от переменной $k$.

Фактически, данная формула задает ограничение, что родитель состояния $d_{j + 1}$ автомата $\mathcal{D}$ имеет номер не меньший, чем номер родителя состояния $d_{j}$. 
Также можно заметить, что учитывая ограничение, заданное формулой~\eqref{eq:p-sum-eq-one}, двоичное число $\mathbf{p_{j}}=\overline{p_{j,1}p_{j,2}\ldots p_{j,j-1}}$ состоит из $j - 2$ нулей и одной единицы.
Тогда рассматриваемая формула говорит, что в числе $\mathbf{p_{j + 1}}$ единственная единица стоит не левее чем в векторе $\mathbf{p_{j}}$ (и наоборот, что в числе $\mathbf{p_{j}}$ единственная единица стоит не правее чем в векторе $\mathbf{p_{j + 1}}$). 
Для удобства сравнения данных чисел, рассматривается расширенное число $\tilde{\mathbf{p_{j}}} = \overline{p_{j,1}p_{j,2}\ldots p_{j,j-1}0}$.
В контексте имеющейся формулы расширенное число не отличается от обычного, так как в нем все еще содержится ровно одна единица на том же месте, что и раньше (считая, слева), но теперь двоичные числа $\tilde{\mathbf{p_{j}}}$ и $\mathbf{p_{j+1}}$ имеют одинаковое число цифр.
Тогда исходная формула фактически задает ограничение $\tilde{\mathbf{p_{j}}} \geq \mathbf{p_{j + 1}}$.
С помощью $\mathbf{p_{j}}^{\mathbf{i}}$ далее в данном разделе будет обозначаться двоичное число, являющееся суффиксом числа $\mathbf{p_{j}}$, начинающимся с $i$-ой цифры и заканчивая последней: $\mathbf{p_{j}}^{\mathbf{i}}=\overline{p_{j,i}p_{j,i+1}\ldots p_{j,j - 1}}$.
Аналогично, $\tilde{\mathbf{p_{j}}}^{\mathbf{i}}=\overline{p_{j,i}p_{j,i+1}\ldots p_{j,j}}$.

Для сравнения необходимо ввести новые булевы переменные $\{\mathit{geq}_{j,i}\}_{1 < j \leq M, 1 \leq i \leq j + 1}$.
Переменная $\{\mathit{geq}_{j,i}\}$ истинна тогда и только тогда, когда число $\tilde{\mathbf{p_{j}}}^{\mathbf{i}}$ больше или равно (\textbf{g}reater or \textbf{eq}ual) чем число $\mathbf{p_{j + 1}}^{\mathbf{i}}$, а значит и единица во втором числе находится не левее чем в первом. 
Определить переменные $\mathit{geq}_{j,i}$ можно рекурсивно следующим образом:
%
\begin{equation}
\label{eq:geq-def}
  \mathit{geq}_{j,i} \leftrightarrow 
    \begin{cases} 
      1                               & i = j + 1, 1 \leq j \leq M \\
      \mathit{geq}_{j,i + 1} \wedge \left(p_{j,i} \leftrightarrow p_{j + 1, i}\right) \vee p_{j,i} \wedge \neg p_{j + 1, i}  & 1 \leq i \leq j \leq M
    \end{cases} 
\end{equation}

Ограничение $\mathit{geq}_{j,j + 1} = 1$ задает исходное значение для рекурсивного определения переменных $\mathit{geq}_{j,i}$.
Далее, число $\tilde{\mathbf{p_{j}}}^{\mathbf{i}}$ больше или равно числа $\mathbf{p_{j + 1}}^{\mathbf{i}}$, то есть $\mathit{geq}_{j,i} = 1$, если $i$-ый бит чисел $\tilde{\mathbf{p_{j}}}$ и $\mathbf{p_{j + 1}}$ совпадает, а для суффиксов $\tilde{\mathbf{p_{j}}}^{\mathbf{i + 1}}$ и $\mathbf{p_{j + 1}}^{\mathbf{i + 1}}$ верно, что первый больше либо равен второго, или если $i$-ый бит числа $\tilde{\mathbf{p_{j}}}$ равен единице, а числа $\mathbf{p_{j + 1}}$~{---} нулю.
Последнее верно, так как оба числа содержат по одной единице и вне зависимости от того, где находится единица во втором числе, правее или левее, число $\tilde{\mathbf{p_{j}}}^{\mathbf{i}}$ строго больше числа $\mathbf{p_{j + 1}}^{\mathbf{i}}$.

Для упрощения записи и уменьшения размера дизъюнктов можно ввести еще одни вспомогательные переменные $\{\mathit{peq}_{j,i}\}_{1 \leq i \leq j \leq M}$.
Переменная $\mathit{peq}_{j,i}$ истинна тогда и только тогда, когда $p_{j,i} = p_{j + 1,i}$.
%
\begin{equation}
\label{eq:peq-def}
  \mathit{peq}_{j,i} \leftrightarrow \left(p_{j,i} \leftrightarrow p_{j + 1, i}\right) 
\end{equation}


Тогда формула~\eqref{eq:geq-def} примет окончательный вид:
%
\begin{equation}
\label{eq:geq-def2}
  \mathit{geq}_{j,i} \leftrightarrow 
    \begin{cases} 
      1                               & i = j + 1, 1 \leq j \leq M \\
      \mathit{geq}_{j,i + 1} \wedge \mathit{peq}_{j,i} \vee p_{j,i} \wedge \neg p_{j + 1, i}  & 1 \leq i \leq j \leq M
    \end{cases} 
\end{equation}
%
При преобразовании в КНФ формула~\eqref{eq:geq-def2} будет состоять из $\mathcal{O}\left(M^{2}\right)$ дизъюнктов.

Используя новые переменные $\mathit{geq}_{i,j}$ формулу~\eqref{eq:p-order} можно переписать следующим образом.
%
\begin{equation}
\label{eq:p-order-tight}
  \bigwedge_{1 < j \leq M} \mathit{geq}_{j,1}
\end{equation}
%
Действительно, $\mathit{geq}_{j,1} = 1 \Leftrightarrow \tilde{\mathbf{p_{j}}} = \tilde{\mathbf{p_{j}}}^\mathbf{1} \geq \mathbf{p_{j + 1}}^{\mathbf{1}} = \mathbf{p_{j + 1}}$.

Таким образом, формула~\eqref{eq:p-def-tight} выражается через $\mathcal{O}\left(M\right)$ дизъюнктов, а формулы~\eqref{eq:peq-def} и~\eqref{eq:geq-def2}~{---} через $\mathcal{O}\left(M^{2}\right)$ дизъюнктов.

%----------------------------------------------------------------------------------------

\subsection{Определение переменных минимального символа}
\label{sec:space:tight:m-def}

Формула $$\bigwedge_{1 \leq i < j \leq M} \bigwedge_{1 \leq n \leq L} \left(m_{i,l_{n},j} \leftrightarrow y_{i,l_{n},j} \wedge \neg y_{i,l_{n - 1}, j} \wedge \neg y_{i,l_{n - 2}, j} \wedge \ldots \wedge \neg y_{i,l_{1},j} \right)$$ при преобразовании в КНФ выражается через $\mathcal{O}\left(M^{2} \times L^{2}\right)$ дизъюнктов, так как обе переменные $i$ и $j$ имеют область допустимых значений размера $M$, переменная $n$~{---} размера $L$, а также правая часть формулы имеет длину $O\left(L\right)$.
Так как переменные $i$, $j$ и $n$ независимы,то, чтобы сократить количество дизъюнктов, нужно сократить правую часть формулы.
Можно заметить, что данная формула по своей структуре аналогична той, что рассматривалась в разделе~\ref{sec:space:tight:p-def}.
Тогда аналогично можно ввести новые булевы переменные $\{\mathit{fy}_{i,l_{n},j}\}_{0 \leq i < j \leq M,0 \leq n \leq M}$.
Переменная $\mathit{fy}_{i,l_{n},j}$ истинна тогда и только тогда, когда все переменные $y_{i,l_{k},j}$, где $1 \leq k \leq n$, ложны (\textbf{f}alse $\boldsymbol{y}$ variables~{---} ложные переменные $y$).
Иными словами, $\mathit{fy}_{i,l_{n},j} \leftrightarrow \neg y_{i,l_{n},j} \wedge \neg y_{i, l_{n - 1}, j} \wedge \ldots \neg y_{i,l_{1},j}$. 
Для $n = 0$ в явном виде определим, что $\mathit{fy}_{i,l_{0},j} = 1$ для любых $i,j$.
Далее, аналогично тому, как это было сделано в разделе~\ref{sec:space:tight:p-def}, определим переменные $\mathit{fy}_{i,l_{n},j}$ рекурсивно.

\begin{equation}
\label{eq:fy-def}
  \mathit{fy}_{i,l_{n},j} \leftrightarrow 
    \begin{cases} 
      1                               & 1 \leq i < j \leq M, n = 0 \\
      \mathit{fy}_{i,l_{n - 1},j} \wedge \neg y_{i,l_{n},j}  & 1 \leq i < j \leq M, 1 \leq n \leq L
    \end{cases} 
\end{equation}
%
При преобразовании в КНФ формула~\eqref{eq:fy-def} будет состоять из $\mathcal{O}\left(M^{2} \times L\right)$ дизъюнктов.

Используя новые переменные $\mathit{fy}_{i,l_{n},j}$ формулу~\eqref{eq:m-def} можно переписать следующим образом.
%
\begin{equation}
\label{eq:m-def-tight}
  \bigwedge_{1 \leq i < j \leq M} \bigwedge_{1 \leq n \leq L} \left(m_{i,l_{n},j} \leftrightarrow y_{i,l_{n},j} \wedge \mathit{fy}_{i,l_{n - 1},j} \right)
\end{equation}
%
В данной формуле решена проблема длинной правой части и общее число дизъюнктов, требуемых для кодирования переменных $m_{i,l_{n},j}$, также $\mathcal{O}\left(M^{2} \times L\right)$. 

%----------------------------------------------------------------------------------------

\subsection{Порядок детей одного родителя}
\label{sec:space:tight:m-order}

Формула $$\bigwedge_{1 \leq i < j \leq M} \bigwedge_{1 \leq k < n \leq L} \left(p_{j,i} \wedge p_{j + 1, i} \wedge m_{i,l_{n}, j} \rightarrow \neg m_{i, l_{k}, j + 1}\right)$$ при преобразовании в КНФ выражается также через $\mathcal{O}\left(M^{2} \times L^{2}\right)$ дизъюнктов, так как обе переменные $i$ и $j$ имеют область допустимых значений размера $M$, а обе переменные $n$ и $k$~{---} размера $L$.
Переменные $i$,$j$ и $n$ являются независимыми, поэтому сократить размер всей формулы можно только избавившись от переменной $k$.
Можно заметить, что данная формула по своей структуре аналогична той, что рассматривалась в разделе~\ref{sec:space:tight:p-order}.
Тогда аналогично можно ввести новые булевы переменные $\{\mathit{fm}_{i,l_{n},j}\}_{0 \leq i < j \leq M,0 \leq n \leq M}$.
Переменная $\mathit{fm}_{i,l_{n},j}$ истинна тогда и только тогда, когда все переменные $m_{i,l_{k},j}$, где $1 \leq k \leq n$, ложны (\textbf{f}alse $\boldsymbol{m}$ variables~{---} ложные переменные $m$).
Иными словами, $\mathit{fm}_{i,l_{n},j} \leftrightarrow \neg m_{i,l_{n},j} \wedge \neg m_{i, l_{n - 1}, j} \wedge \ldots \neg m_{i,l_{1},j}$. 
Для $n = 0$ в явном виде определим, что $\mathit{fm}_{i,l_{0},j} = 1$ для любых $i,j$.
Далее, аналогично тому, как это было сделано в разделе~\ref{sec:space:tight:p-order}, определим переменные $\mathit{fm}_{i,l_{n},j}$ рекурсивно.

\begin{equation}
\label{eq:fm-def}
  \mathit{fm}_{i,l_{n},j} \leftrightarrow 
    \begin{cases} 
      1                               & 1 \leq i < j \leq M, n = 0 \\
      \mathit{fm}_{i,l_{n - 1},j} \wedge \neg m_{i,l_{n},j}  & 1 \leq i < j \leq M, 1 \leq n \leq L
    \end{cases} 
\end{equation}
%
При преобразовании в КНФ формула~\eqref{eq:fm-def} будет состоять из $\mathcal{O}\left(M^{2} \times L\right)$ дизъюнктов.

Используя новые переменные $\mathit{fm}_{i,l_{n},j}$ формулу~\eqref{eq:m-order} можно переписать следующим образом.
%
\begin{equation}
\label{eq:m-order-tight}
  \bigwedge_{1 \leq i < j < M} \bigwedge_{1 \leq n \leq L} \left(p_{j,i} \wedge p_{j + 1, i} \wedge m_{i,l_{n}, j} \rightarrow \neg \mathit{fm}_{i, l_{n - 1}, j + 1}\right)
\end{equation}
%
Таким образом общее число дизъюнктов, требуемых для ограничения порядка детей одного состояния, равняется $\mathcal{O}\left(M^{2} \times L\right)$. 

%------------------------------------------------------------------------------------------------------

\section{Подходы к сокращению пространства поиска, основанные на структурных особенностях автомата} 
\label{sec:space:pruning}

В данной главе предлагаются новые методы по сокращению пространства поиска в задаче генерации ДКА минимального размера по заданным словарям. 
Данные методы не являются необходимыми для нахождения соответствующего автомата, но помогают сделать это быстрее.
В основе предлагаемых методов лежит использование структурных особенностей BFS пронумерованного ДКА, а также связь между расширенным префиксным деревом и ДКА.

\subsection{Полное дерево обхода в ширину}
\label{sec:space:pruning:bfs-tree}

\inote{рисунок дерева}

На рисунке \inote{ссылка} показано полное BFS дерево, построенное по некоторому автомату.
Данное дерево является полным, так как у каждой его внутренней вершины имеется по $L$ детей.
Тогда данное дерево показывает максимально возможные номера, которые могут быть у детей некоторого состояния $d_{i}$.
Действительно, нельзя добавить в данное дерево новые вершины, которые будут иметь номер между $i$ и $i \cdot L + 1$, так как все возможные позиции заняты.
В то же время, если удалить какие-то из вершин правее или ниже вершины $d_{i}$, то номера детей могут только уменьшиться.
Далее будут представлены дополнительные ограничения, которые следуют из рисунка \inote{ссылка}.

\paragraph{Сокращение области определения родительских переменных.}
У некоторого состояния $d_{i}$, где $1 \leq i < M$, детьми в BFS дереве могут быть только состояния с номерами от $i + 1$ до $\min\left(i \cdot L + 1, M\right)$.
Так как в BFS дереве номер ребенка всегда больше номера родителя, то нижняя граница тривиальна.
Рисунок \inote{ссылка} иллюстрирует обоснование верхней границы.
Действительно, можно доказать по индукции, что состояния на $k$-ом уровне имеют номера от $\sum_{r = 0}^{k - 1}L^{r} + 1$ до $\sum_{r = 0}^{k}L^{r}$.
База индукции при $k = 0$, очевидно, верна.
Если для некоторого слоя $k$ утверждение выше верно, то для слоя $k + 1$ верно, что нумерация состояний на нем начинается с $\sum_{r = 0}^{k - 1}L^{r} + 1$, а всего вершин $\left(\sum_{r = 0}^{k}L^{r} - \left(\sum_{r = 0}^{k - 1}L^{r} + 1\right) + 1\right) \cdot L = L^{k} * L = L^{k + 1}$, из чего следует, что последнее состояние имеет номер $\sum_{r = 0}^{k}L^{r} + L^{k + 1} = \sum_{r = 0}^{k + 1}L^{r}$.

\inote{возможно, оформить в виде полноценной теоремы и доказательства.}

Выразить данное свойство можно, либо определив переменные для соответствующей области определения~{---} $\{p_{j,i}\}_{1 \leq i < j \leq min(i \cdot L + 1, M)}$, либо в явном виде указав, что $p_{j,i} = 0$ при $j > i \cdot L + 1$.

\paragraph{Сокращение области определения переменных перехода и переменных наличия переходов.}
Помимо закономерностей между номерами родителей и детей в BFS пронумерованном автомате, можно заметить более общую закономерность относительно переходов.
Из состояния $d_{i}$ в BFS пронумерованном автомате не может в принципе существовать перехода в состояние $d_{j}$ если $j > i \cdot L + 1$.
Действительно, из доказанного в предыдущей секции следует, что у состояния $d_{j}$ родителем должно быть состояние $d_{k}$, где $k > i$.
Но, если существует из состояния $d_{i}$ существует переход в состояние $d_{j}$, то по принципу BFS обхода родителем состояния $d_{j}$ должно быть состояние $d_{k}$, где $k \leq i$.
Получившееся противоречие доказывает исходное утверждение.
Таким образом, можно сделать заключение, что $y_{i,l,j} = 0$ при $j > i \cdot L + 1; l \in \Sigma$.

Как следствие, по определению переменных наличия переходов верно, что $t_{i, j} = 0$ при $j > i \cdot L + 1$.

\inote{$y_{i,l,iL+2-j}$ --- пока скипнул, может добавить про них}

%----------------------------------------------------------------------------------------

\subsection{Свойство непрерывности родительских переменных}
\label{sec:space:pruning:continuity}

Помимо того, что у каждого состояния $d_{i}$ автомата $\mathcal{D}$ детьми могут быть состояния с номерами от $i + 1$ до $i \cdot L + 1$, можно утверждать, что состояние $d_{i}$ может быть родителем не более чем $L$ состояний, которые при этом пронумерованны последовательно.
Количество детей ограничено размером алфавита, так как рассматриваемый автомат является детерминированным.
Последовательная нумерация следует из структуры алгоритма BFS~{---} дети некоторого состояния поочередно добавляются в очередь и им присваиваются последовательные номера.
Данное свойство можно назвать \emph{свойством непрерывности}.
Для булевого кодирования предикатов нарушения симметрии данное свойство означает, что для фиксированного $i$ переменные $p_{j,i}$ ложны для всех $j$, кроме некоторого отрезка $[j_{0},\ldots,j_{s}]$, где $1 \leq j_{0} \leq j_{s} \leq M, s\leq L$.

Можно добавить дополнительные ограничения, задающие данное свойство, которые дополнительно ограничат пространство поиска.
Для этого необходимо ввести два дополнительных множества булевых переменных~{---} $\{\mathit{lnp}_{j,i}\}_{1 \leq i < j \leq M}$ и  $\{\mathit{rnp}_{j,i}\}_{1 \leq i < j \leq M}$.

Переменная $\mathit{lnp}_{j,i}$ истинна тогда, когда переменная $p_{j,i} = 0$ и $j < j_{0}$.
Иными словами, данная переменная истинна в случае, когда $j$ находится левее отрезка истинных родительских переменных (\textbf{l}eft \textbf{n}o \textbf{p}arent).
\inote{рисунок}.
Определить на языке выполнимости булевых формул данные переменные можно следующим образом.
Формула $$\bigwedge_{1 \leq i < j \leq M} \neg p_{j,i} \wedge p_{j + 1, i} \rightarrow \mathit{lnp}_{j,i}$$ задает пограничное истинное значение переменных $\mathit{lnp}_{j,i}$.
Далее, необходимо добавить формулу $$\bigwedge_{1 \leq i < M, i + 1 < j \leq M} \mathit{lnp}_{j,i} \rightarrow \mathit{lnp}_{j - 1, i},$$
которая задает значения переменных $\mathit{lnp}_{j,i}$ левее пограничного.
Как следствие из определения переменных $\mathit{lnp}_{j,i}$, можно добавить следующую формулу: $$\bigwedge_{1 \leq i < j \leq M} \mathit{lnp}_{j,i} \rightarrow \neg p_{j,i}.$$
Таким образом, переменные $\mathit{lnp}_{j,i}$ для каждого $i$ истинны начиная с $j = 1$ и до тех пор, пока $p_{j + 1, i}$ не будет истинно.
Можно заметить, что начиная с момента, когда $p_{j,i}$ истинно, значение переменных $\mathit{lnp}_{j,i}$ не определено, что, как будет показано далее, не играет никакой роли.

Аналогичным образом определяются переменные $\mathit{rnp}_{j,i}$.
Переменная $\mathit{rnp}_{j,i}$ истина тогда, когда переменная $p_{j,i} = 0$ и $j > j_{s}$, то есть когда $j$ находится правее отрезка истинных родительских переменных (\textbf{r}ight \textbf{n}o \textbf{p}arent).
Пограничное истинное значение переменных $\mathit{rnp}_{j,i}$ задается с помощью формулы $$\bigwedge_{1 \leq i < j \leq M} p_{j - 1,i} \wedge \neg p_{j, i} \rightarrow \mathit{rnp}_{j,i}.$$
Значение переменных правее пограничного задаются аналогично предыдущему случаю: $$\bigwedge_{1 \leq i < j < M} \mathit{rnp}_{j,i} \rightarrow \mathit{rnp}_{j + 1, i}.$$
Как и в случае с переменными $\mathit{lnp}_{j,i}$, можно добавить формулу $$\bigwedge_{1 \leq i < j \leq M} \mathit{rnp}_{j,i} \rightarrow \neg p_{j,i}.$$
Переменные $\mathit{rnp}_{j,i}$ для каждого $i$ истинны начиная с $j = M$ и в порядке убывания истинны до тех пор, пока $p_{j - 1, i}$ не будет истинно.
Можно заметить, что, аналогично, начиная с $j = 1$, и до тех пор, пока $p_{j,i}$ не станет ложной после серии истинных значений, значение переменных $\mathit{rnp}_{j,i}$ не определено.
Помимо этого, можно добавить следующую формулу: $$\bigwedge_{1 \leq i < j \leq M, l \in \Sigma} \mathit{rnp}_{j,i} \rightarrow \neg y_{i,l,j}.$$
Действительно, если состояние $d_{i}$ имеет детей с номерами $j_{0},\ldots,j_{s}$, то из состояния $d_{i}$ не может быть переходов состояния с номерами б\emph{о}льшими чем $j_{s}$, иначе данные состояния были бы также детьми состояния $d_{i}$. 

Дополнительно, из того, что $d_{i}$ может иметь не более чем $L$ детей, следует, что $$\bigwedge_{1 \leq i < M; i + L < j \leq M - L} p_{j,i} \rightarrow \mathit{lnp}_{j - L, i}$$ и что $$\bigwedge_{1 \leq i < j \leq M - L} p_{j,i} \rightarrow \mathit{rnp}_{j + L, i}.$$

Переменные $\mathit{lnp}_{j,i}$ и $\mathit{rnp}_{j,i}$ помогают задать некоторым переменным $p_{j,i}$ ложное значение.
Однако, исходя из их значения, можно некоторым переменным $p_{j,i}$ задать истинное значение.
Так, если для некоторых $j_{1} < j_{2}$ верно, что $\mathit{lnp}_{j_{1}, i}$ и $\mathit{rnp}_{j_{2}, i}$ ложны, то для всех $j'$ таких, что $j_{1} \leq j' \leq j_{2}$ верно, что $p_{j',i}$ истинна.
Формально, $$\bigwedge_{1 \leq i < M;i < j_{1} \leq j' \leq j_{2} \leq \min\left(j_{1} + L - 1, M\right)} \neg \mathit{lnp}_{j_{1},i} \wedge \neg \mathit{rnp}_{j_{2},i} \rightarrow p_{j',i}.$$

Также, учитывая, что дети некоторого состояния $d_{i}$ пронумерованны последовательно, можно добавить следующее ограничение: $$\bigwedge_{1 \leq i < j < k \leq \min(j + L - 1, M)} p_{j,i} \wedge p_{k,i} \rightarrow p_{k - 1, i}.$$ 

%----------------------------------------------------------------------------------------

\subsection{Минимальное расстояние в дереве обхода автомата в ширину}
\label{sec:space:pruning:bfs-distance}

Еще одним следствием анализа полного BFS дерева \inote{рисунок}, является ограничение минимального расстояния от стартового состояния автомата $\mathcal{D}$ до всех других.
Не сложно заметить, что в полном BFS дереве, представленном на \inote{рисунке}, глубина некоторого состояния $d_{j}$ минимальна.
Действительно, в неполном BFS дереве на каждой глубине состояний не больше чем в полном дереве, а значит состояние может находится или на том же уровне, или глубже. 
Тогда глубина состояния в полном BFS дереве будет являться оценкой снизу для глубины состояния в случайном дереве.

Для доказательства можно воспользоваться ранее доказанным фактом, что на уровне $k$ в полном BFS дереве находятся состояния с номерами от $\left(\sum_{i = 0}^{k - 1} + 1\right)$ до $\left(\sum_{i = 0}^{k}\right)$.
Иными словами, номера состояний на уровне $k$ находятся в полуоткрытом интервале $\left(\sum_{i = 0}^{k - 1}L^{i};\sum_{i = 0}^{k}L^{i}\right]$.
Если домножить левую и правую границы интервала на $(L - 1)$ и прибавить единицу, то получится интервал $\left(L^{k};L^{k + 1}\right]$.
Теперь, если взять логарифм по основанию $L$ от обеих границ и вычесть единицу, то получится, интервал $\left(k - 1; k\right]$.
Из этого можно заключить, что минимальная глубина состояния с номером $j$, а значит и минимальное расстояние от стартового состояния до него, равняется $D_{\min}\left(j\right) = \ceil*{\log_{L}\left(j \cdot \left(L - 1\right) + 1\right) - 1}$.

Таким образом, для любого состояния $d_{j}$ автомата $\mathcal{D}$ минимальное расстояние от стартового состояния $d_{1}$ до $d_{j}$ не меньше, чем $D_{\min}\left(j\right)$. 
Тогда, если расстояние от корня $t_{1}$ префиксного дерева $\mathcal{T}$ до некоторой вершины $t_{v}$ меньше, чем минимально возможное расстояние до состояния $d_{j}$ автомата $\mathcal{D}$: $\Delta\left(v\right) < D_{\min}\left(j\right)$, то можно утверждать, что вершина $t_{v}$ не может соответствовать состоянию $d_{j}$, то есть $x_{v,j} = 0$.

%----------------------------------------------------------------------------------------

\subsection{Дополнительные ограничения, основанные на связи между префиксным деревом и автоматом}
\label{sec:space:pruning:apta-exploiting}

\inote{возможно, добавить этот раздел, пока хз}

%----------------------------------------------------------------------------------------

\section{Реализация и экспериментальные исседования методов, использующих разработанные подходы к сокращению пространства поиска}
\label{sec:space:results}

%----------------------------------------------------------------------------------------

\subsection{Программное средство для генерации детерминированных конечных автоматов по примерам поведения}
\label{sec:space:results:dfa-inductor-py}

%----------------------------------------------------------------------------------------

\subsection{Реализация разработанных методов генерации детерминированных конечных автоматов}
\label{sec:space:results:impl}

%----------------------------------------------------------------------------------------

\subsection{Экспериментальные исследования метода, использующего предикаты нарушения симметрии на основе алгоритма обхода графа в глубину}
\label{sec:space:results:dfs}

%----------------------------------------------------------------------------------------

\subsection{Экспериментальные исследования метода, использующего компактные предикаты нарушения симметрии}
\label{sec:space:results:tight}

%----------------------------------------------------------------------------------------

\subsection{Экспериментальные исследования метода, использующего структурные особенности автомата}
\label{sec:space:results:pruning}

%----------------------------------------------------------------------------------------


\chresults{\ref{sec:space}}