%!TEX root = ../dissertation.tex

\chapter{Обзор предметной области} \label{ch:review}

\section{Детерминированные конечные автоматы} \label{sec:dfa-def}

\emph{Алфавитом} $\Sigma$ называется некоторое конечное непустое множество символов.
\emph{Размером алфавита} $\Sigma$ называется число его символов~{---} $L = \abs{\Sigma}$.
В данной диссертации в основном будет рассматривать бинарный алфавит $\Sigma = \mathbb{B} = \{0, 1\}$.
\emph{Словом} (\emph{строкой}, \emph{цепочкой}) $\omega$ называется конечная последовательность символов некоторого алфавита. 
\emph{Длиной слова} называется число символов в нем, обозначается как $\abs{\omega}$.
Множество слов длины $k$ над алфавитом $\Sigma$ обозначается как $\Sigma^{k}$.
\emph{Пустой строкой} называется слово, не содержащее ни одного символа.
Такая строка, обозначаемая как $\varepsilon$, имеет нулевую длиной и может рассматриваться как слово над любым алфавитом~{---} $\Sigma^{0}=\{\varepsilon\}$.
Множество всех возможных слов, составленных из символов некоторого алфавита $\Sigma$, является его замыканием:
$$\Sigma^{*} = \bigcup_{k=0}^{\infty}\Sigma^{k}.$$
Подмножество множества всех слов над алфавитом $\Sigma$ называется \emph{языком}~--- $\mathcal{L} \subset \Sigma^{*}$.

\emph{Детерминированным конечным автоматом} (ДКА) называется пятерка $\mathcal{D} = \left(D,\Sigma,\delta,d_{1},D^{+}\right)$, где $D$~{---} конечное множество состояний, $\Sigma$~{---} алфавит входных символов, $\delta:D \times \Sigma \rightarrow D$~{---} \emph{функция переходов}, $d_{1}$~{---} \emph{стартовое} (\emph{начальное}) состояние, $D^{+} \subset D$~{---} множество \emph{допускающих} (\emph{принимающих}) состояний. 
В неявном виде также задано множество \emph{недопускающих} (\emph{отвергающих}) состояний $D^{-} = D \setminus D^{+}$.

Индуктивно определим вспомогательную \emph{расширенную функцию переходов} $\hat{\delta}: D \times \Sigma^{*} \rightarrow D$:
\begin{enumerate}
  \item для любого состояния $d_{i}$ верно, что переход по пустой строке не осуществляется~{---} $\hat{\delta}\left(d_{i},\varepsilon\right) = d_{i}$;
  \item для любого состояния $d_{i}$ верно, что переход по строке $\pi = \pi'c$, где $\pi,\pi' \in \Sigma^{*}$, $c \in \Sigma$, может быть определен следующим образом $\hat{\delta}\left(d_{i}, \pi\right) = \delta\left(\hat{\delta}\left(d_{i}, \pi'\right), c\right)$.
\end{enumerate}
Говорят, что ДКА $\mathcal{D}$ \emph{допускает} (\emph{принимает}) слово $\omega$, если $\hat{\delta}\left(d_{1}, \omega\right) \in D^{+}$. 
Иначе, если $\hat{\delta}\left(d_{1}, \omega\right) \in D^{-}$, говорят, что ДКА $\mathcal{D}$ \emph{не допускает} (\emph{отвергает}) слово $\omega$. 
Множество всех слов, допускаемых автоматом $\mathcal{D}$: $\mathcal{L}\left(D\right) = \{\omega \mid \hat{\delta}\left(d_{1}, \omega \right)\}$, называется языком автомата $\mathcal{D}$.
