%!TEX root = ../dissertation.tex
\begin{enumerate}
  \item Разработаны предикаты нарушения симметрии, основанные на алгоритмах обхода графа в ширину и в глубину, для сокращения пространства поиска при решении задачи выполнимости и точные методы генерации ДКА по заданным примерам поведения, использующие данные предикаты. 
  Предикаты нарушения симметрии на алгоритма обхода графа в глубину не продемонстрировали никаких преимуществ, относительно предикатов, основанных на алгоритме обхода графа в ширину.
  Новый способ кодирования предикатов нарушения симметрии на основе обхода графа в ширину выражается с помощью асимптотически меньшего числа дизъюнктов.
  Разработанный точный метод, использующие данные предикаты вкупе с предикатами, кодирующими некоторые особенности дерева обхода графа в ширину, демонстрирует лучшую производительность относительно существовавших ранее методов.

  \item Разработан точный метод генерации ДКА по избыточному набору примеров поведения с использованием сведения к задаче выполнимости и подхода уточнения абстракции по контрпримерам.
  Размер булевой формулы и число используемых переменных линейно возрастает при увеличении числа примеров поведения.
  Разработанный метод позволяет генерировать ДКА по избыточному набору примеров поведения, итеративно расширяя множество используемых примеров, достраивая расширенное префиксное дерево и используя программное средство для решения SAT в инкрементальном режиме.
  Так как множество используемых примеров поведения расширяется контрпримерами к построенным промежуточным ДКА, то в итоге для генерации используются только значимые примеры поведения.
  Таким образом, с помощью данного метода удается точно решать такие примеры задач, которые раньше не могли быть решены ввиду большой булевой формулы.

  \item Разработан метод генерации всех неизоморфных ДКА минимального размера, удовлетворяющих заданным примерам поведения, с использованием предикатов нарушения симметрии и программных средств решения задачи выполнимости.
  Ранее задача генерации всех различных ДКА минимального размера не имела эффективного решения ввиду того, что изоморфные автоматы, являясь одинаковыми по структуре и по определяемому языку, программным средством для решения SAT считаются различными, что приводит к рассмотрению $\mathcal{O}\left(M!\right)$ изоморфных автоматов размер $M$.
  Использование предикатов нарушения симметрии на основе алгоритма обхода графа в ширину позволяет для каждого класса эквивалентности по изоморфизму оставить для рассмотрения единственного представителя вместо факториала. 
  Таким образом, впервые был предложен метод решения поставленной задачи.
  Поиск всех различных автоматов может быть полезен для дальнейшего их анализа, либо для анализа имеющихся примеров поведения.
  Другим потенциальным применением является возможность доказать единственность минимального автомата, соответствующего заданным примерам поведения.

 \end{enumerate}

\inote{возможно, что-то еще дописать. Воды, рекомендации, задела на будущее, внедрение и благодарности}