%!TEX root = ../dissertation.tex
\begin{enumerate}
  \item Разработаны предикаты нарушения симметрии, основанные на кодировании алгоритмов обхода графа в ширину и в глубину, для сокращения пространства поиска при решении задачи выполнимости и точные методы генерации ДКА по заданным примерам поведения, использующие данные предикаты. 
  Предикаты нарушения симметрии, основанные на кодировании алгоритма обхода графа в глубину имеют скорее теоретическую значимость, так как не продемонстрировали никаких преимуществ, относительно предикатов, основанных на кодировании алгоритма обхода графа в ширину.
  Новый способ кодирования предикатов нарушения симметрии на основе кодирования алгоритма обхода графа в ширину позволяет использовать асимптотически меньшее число дизъюнктов относительно предложенного ранее.
  Разработанный точный метод, использующий данные предикаты совместно с предикатами, учитывающими особенности дерева обхода графа в ширину, демонстрирует лучшую производительность относительно существовавших ранее методов.

  \item Разработан точный метод генерации ДКА по избыточному набору примеров поведения с использованием сведения к задаче выполнимости и подхода уточнения абстракции по контрпримерам.
  Размер булевой формулы и число используемых переменных линейно возрастает при увеличении числа примеров поведения.
  Разработанный метод позволяет генерировать ДКА по избыточному набору примеров поведения, итеративно расширяя множество используемых примеров, достраивая расширенное префиксное дерево и используя программное средство для решения SAT в инкрементальном режиме.
  Так как множество используемых примеров поведения расширяется контрпримерами к построенным промежуточным ДКА, то в итоге для генерации используются только значимые примеры поведения.
  Таким образом, с помощью данного метода удается точно решать такие задачи, которые раньше не могли быть решены ввиду большого размера булевой формулы.

  \item Разработан метод генерации всех неизоморфных ДКА минимального размера, удовлетворяющих заданным примерам поведения, с использованием предикатов нарушения симметрии и программных средств решения задачи выполнимости.
  Ранее задача генерации всех неизоморфных ДКА минимального размера не имела эффективного решения ввиду того, что изоморфные автоматы, являясь одинаковыми по структуре и по определяемому языку, программным средством для решения SAT считаются различными, что приводит к рассмотрению $\mathcal{O}\left(M!\right)$ изоморфных автоматов с $M$ состояниями.
  Использование предикатов нарушения симметрии на основе кодирования алгоритма обхода графа в ширину позволяет для каждого класса эквивалентности по изоморфизму оставить для рассмотрения единственного представителя вместо факториала. 
  Таким образом, впервые был предложен метод решения задачи генерации всех неизомофрных ДКА минимального размера.
  Поиск всех неизоморфных автоматов может быть полезен для дальнейшего их анализа, либо для анализа имеющихся примеров поведения.
  Другим применением разработанного метода является возможность доказать единственность минимального автомата, соответствующего заданным примерам поведения.

  \item Во время работы над диссертацией на языке \emph{Python} было разработано программное средство с открытым исходным кодом \texttt{DFA-Inductor-py}, предназначенное для генерации ДКА по заданным примерам поведения.
  В состав средства входят различные модули, позволяющие решать задачу генерации ДКА по заданным примерам поведения, задачу генерации ДКА по избыточному набору примеров поведения и задачу генерации всех неизоморфных ДКА по заданным примерам поведения.
  В средстве реализованы различные предикаты нарушения симметрии~--- как предложенные ранее другими авторами, так и разработанные в рамках настоящей диссертации.

  \item Результаты работы были использованы в учебном процессе на факультете информационных технологий и программирования Университета ИТМО в рамках курса «Проектирование автоматных программ» программы бакалавриата «Математические модели и алгоритмы в разработке программного обеспечения», что подтверждается актом об использовании.

  \item Часть результатов работы использовалась при выполнении проекта SAUNA (``Integrated safety assessment and justification of nuclear power plant automaton''), выполненного исследовательской группой ``IT in Automation'' кафедры электротехники и автоматики университета Аалто, Финляндия, в рамках Финской программы исследований безопасности атомных электростанций~--- SAFIR2018, что подтверждается письмом руководителя исследовательской группы ``IT in Automation'' В.~В. Вяткина.

  \item Результаты работы также использовались при выполнении под руководством автора диссертации гранта Российского фонда фундаментальных исследований (проект 18­37­00425 «Разработка эффективных методов машинного обучения для построения детерминированных конечных автоматов на основе решения задачи выполнимости», 2018–2020 гг.) и в рамках проектов по программе повышения конкурентоспособности ведущих российских университетов среди ведущих мировых научно-образовательных центров <<5-100>>.

\end{enumerate}

Точный метод генерации ДКА по заданным примерам поведения, использующий предикаты нарушения симметрии на основе алгоритма обхода графа в ширину, демонстрирует лучшую производительность относительно известных ранее методов и позволяет генерировать автоматы большего размера за меньшее время.
Метод генерации ДКА по избыточному набору примеров поведения позволяет генерировать автоматы по таким данным, по которым известные ранее методы не могли построить ДКА в принципе ввиду большого размера булевой формулы.
Метод генерации всех неизоморфных ДКА минимального размера, удовлетворяющих заданным примерам поведения, является первым известным методом, позволяющим сгенерировать все неизоморфные автоматы~--- существовавшие ранее методы могут быть адаптированы для поиска всех ДКА, но их использование приводит к возникновению комбинаторного взрыва числа рассматриваемых автоматов ввиду отсутствия эффективных предикатов нарушения симметрии.
Целью настоящего диссертационного исследования являлось повышение эффективности точных методов генерации детерминированных конечных автоматов по заданным примерам поведения посредством сокращения пространства поиска при решении задачи выполнимости.
Таким образом, согласно результатам экспериментальных исследований, цель можно считать успешно достигнутой.

\paragraph*{Благодарности.}
Автор выражает благодарность своему научному руководителю, кандидату технических наук, В.~И. Ульянцеву за неоценимую помощь в исследовательской деятельности и в написании настоящей работы, профессору, доктору технических наук, А.~А. Шалыто за наставничество, коллегам по международной лаборатории <<Компьютерные технологии>> кандидату технических наук Д.~С. Чивилихину за многочисленные консультации по вопросам исследовательской деятельности и за помощь с переводом текста реферата на английский язык, К.~И. Чухареву и Д.~М. Суворову за помощь в подготовке иллюстративного материала настоящей диссертации, А. И. Бугровскому и Т.~Р. Галимжанову за помощь в документообороте во время процедуры подготовки защиты настоящей диссертационной работы, иностранным коллегам кандидату физико-математических наук А.~И. Игнатьеву и доктору Ж. Маркешу-Сильве за организацию стажировки и активное в ней участие, в рамках которой проводилась часть исследовательской деятельности автора.

Также автор выражает благодарность своим родителям Тимуру и Валентине за привитую любовь к знаниям и математике, Лазаревой Екатерине за то, что всегда была рядом, своим друзьям Александру Б., Александру С., Анастасии, Валентину, Евгению, Марине, Матвею, Наталье, Талгату и Татьяне за оказанную моральную поддержку.