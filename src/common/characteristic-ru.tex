%!TEX root = ../dissertation.tex
\textbf{Актуальность темы.} 
Конечные автоматы являются одним из основополагающих концептов в дискретной математике, информатике и программировании.
Помимо непосредственной роли в теории формальных языков и при проектировании вычислительных машин, различные конечно-автоматные модели используются в наше время на практике при разработке и анализе программного обеспечения.

Например, конечные автоматы используются при проектировании моделей программного обеспечения контроллеров и ответственных систем~\cite{shalyto-automata-2010,DBLP:conf/setta/PatilDV15}, для спецификации протоколов взаимодействия~\cite{DBLP:conf/coordination/JongmansHA14}, для моделирования поведения высокоуровневых систем~\cite{DBLP:journals/ese/HeuleV13,wagner2006modeling}.
Основными преимуществами использования автоматов являются наглядность и относительная простота модели для человеческого восприятия, а также возможность автоматизированной верификации формальных свойств модели (model checking)~\cite{clarke2018model}.

Во многих случаях проектирование автоматной модели осуществляется разработчиком вручную, например, в парадигме автоматного программирования~\cite{shalyto-automata-2010}.
При решении других задач подразумевается автоматизированная генерация конечного автомата~--- извлечение модели из существующих данных или систем.
Среди практических примеров использования методов генерации конечных автоматов можно привести: построение моделей программного обеспечения контроллеров по составленным вручную или снятым автоматизированно примерам поведения~\cite{DBLP:conf/etfa/ChivilikhinBUSS18}, построение формальных моделей объектов управления~\cite{DBLP:conf/etfa/BuzhinskyV17,DBLP:journals/tii/BuzhinskyV17}, анализ моделей поведения сложных программных систем~\cite{DBLP:journals/tosem/CookW98,DBLP:conf/sigsoft/BertolinoIPT09,DBLP:journals/ese/HeuleV13} и сетевых протоколов~\cite{DBLP:conf/sp/SivakornAPKJ17}, и другие.
Активное изучение и разработка алгоритмов генерации автоматов начиналась ведется с 1970-х годов.
В 1978 году было доказано, что задача генерации детерминированного конечного автомата (ДКА) с заданным (минимальным) числом состояний по заданным примерам поведения является NP-полной~\cite{DBLP:journals/iandc/Gold78}.
Данный теоретический результат подчеркивает сложность задачи генерации автоматов в общем случае, актуализируя разработку применимых на практике алгоритмов.

Впоследствии было предложено достаточно много как эвристических, так и метаэвристических алгоритмов генерации ДКА по заданным примерам поведения, и к настоящему моменту они образуют собой целое семейство алгоритмов в дискретной математике.
За последние годы была разработана группа методов, сводящих задачу точной генерации (с минимально возможным числом состояний) искомого автомата к другим NP-полным задачам.
При этом наиболее эффективные подходы в настоящий момент основаны на сведении к задаче выполнимости.

Задача выполнимости булевых формул (Boolean Satisfiability~--- SAT) заключается в определении существования выполняющей подстановки для заданной булевой формулы, представленной в конъюнктивной нормальной форме~--- в виде конъюнкции дизъюнктов.
Согласно теореме Кука-Левина 1971 года, задача выполнимости является NP-полной
Данный факт актуализировал и стимулировал разработку применимых на практике программных средств для решения задачи выполнимости.

Методы решения SAT разрабатывались еще до введения теоретических оценок сложности: в 1962 году был предложен алгоритм Дэвиса-Патнема-Логемана-Лавленда (Davis-Putnam-Logemann-Loveland~--- DPLL)~\cite{DBLP:journals/cacm/DavisLL62}~--- полный алгоритм поиска с возвратом выполняющей подстановки, использующий эвристики распространения переменной и исключения «чистых» переменных для ускорения поиска.
Данный алгоритм, в общем случае за экспоненциальное время от числа переменных, перебирает все пространство поиска подстановок и останавливается, если была найдена выполняющая подстановка или если перебор завершил работу~--- выполняющей подстановки не существует. 

В середине 1990-х на основе алгоритма DPLL был разработан алгоритм CDCL~\cite{DBLP:conf/iccad/SilvaS96} (conflict-driven clause learning~--- «управляемое конфликтами обучение дизъюнктов»), сохраняющий и использующий в дальнейшем выведенные дизъюнкты в ходе анализа конфликтов, возникающих при работе DPLL.
Такие дизъюнкты в дальнейшем позволяют намного раньше принимать решение о невыполнимости формулы с текущими допущениями и переходить к рассмотрению следующих.

Именно на алгоритме CDCL основаны все современные программные средства для решения SAT.
Каждый год проводятся соревнования программных средств решения SAT~\cite{sat-competitions,sat-competition-2020}, что отличает задачу выполнимости от всех остальных NP-трудных задач.
За счет этого для других задач актуально разрабатывать методы, основанные на сведении к SAT: ускорения метода можно добиться без изменения исходного кода, будет достаточно лишь заменить программное средство для решения SAT, выдающее найденную подстановку или доказывающее ее отсутствие.

Однако, в методологии сведения поставленной задачи к SAT фундаментальный характер носит не столько используемое программное средство, сколько способ трансляции экземпляра задачи в булеву формулу, для которой в дальнейшем будет запущен поиск подстановки.
Так, в последние годы для задачи точной генерации ДКА по заданным примерам поведения было предложено базовое сведение~\cite{heule-icgi10} и несколько его модификаций~\cite{DBLP:journals/ese/HeuleV13,ulyantsev-phd-13}.
Данные модификации предлагали использование предикатов нарушения симметрии~--- дополнительно введенных в исходную формулу дизъюнктов, задающих специфичное для задачи сокращение пространства поиска.

Данные техники сокращения пространства поиска оказались очень эффективны на практике, позволив решать задачи генерации автоматов большего размера.
Однако, последующий анализ показывает, что данные техники не оптимальны, а область применимости существующих методов точной генерации ДКА по примерам поведения все еще весьма ограничена (что обусловлено, в первую очередь, NP-полной природой задачи).
Таким образом, тема настоящей диссертации, продолжающей указанные исследования последних лет и направленной на расширение возможностей методов сокращения пространства поиска и генерации ДКА, \textbf{является актуальной}.

\textbf{Степень разработанности темы.}
Задача генерации детерминированного конечного автомата по заданным примерам поведения, выраженным в виде двух множеств $S_{+}$ и $S_{-}$, заключается в поиске ДКА с минимальным числом состояний, такого, что все строки из множества $S_{+}$ принимаются автоматом, а все строки из $S_{-}$~--- не принимаются.
Впервые данная задача была сформулирована в работе Голда в 1967 году~\cite{DBLP:journals/iandc/Gold67}.

Первый известный алгоритм для решения данной задачи был предложен в работе Трахтенброта и Барздиня в 1970 году~\cite{trakhtenbrot-1973-modeling}~--- \texttt{TB}-алгоритм.
Однако, предложенный алгоритм решает только частный случай задачи генерации ДКА по заданным примерам поведения: для некоторого натурального $k$ все возможные слова длины $k$ над алфавитом $\Sigma$~--- всего $\abs{\Sigma}^{k}$ слов~--- должны содержаться во множествах $S_{+}$ и $S_{-}$.
В данном алгоритме, как и в подавляющем большинстве последующих, примеры поведения представляются в виде расширенного префиксного дерева~--- префиксного дерева, в котором вершины могут быть допускающими, отвергающими или промежуточными.
\texttt{TB}-алгоритм основан на полном переборе всех возможных пар состояний префиксного дерева и слиянии эквивалентных состояний в одно.

В 1978 году Голд доказал NP-полноту задачи генерации ДКА заданного, а значит и минимального, размера~\cite{DBLP:journals/iandc/Gold78}.
Ввиду указанной сложности новые алгоритмы генерации минимального ДКА не предлагались более десяти лет.
В последующие годы начали активно разрабатываться неточные эвристические алгоритмы~--- ими не гарантируется минимальность найденного ДКА.
Среди таких алгоритмов можно выделить следующие: \texttt{traxbar}~\cite{DBLP:conf/colt/Lang92}, \texttt{RPNI}~\cite{oncina-rpni-1992}, \texttt{EDSM}~\cite{DBLP:conf/icgi/LangPP98}, \texttt{exbar}~\cite{lang-1999-faster}, \texttt{Windowed-EDSM}~\cite{DBLP:conf/icgi/CicchelloK02}.
Все перечисленные алгоритмы основаны на слиянии состояний расширенного префиксного дерева.
Состояния для слияния выбираются эвристически, чем одновременно объясняется высокая скорость работы и неточность данных алгоритмов.

Другим распространенным подходом к генерации ДКА по заданным примерам поведения является применение метаэвристических алгоритмов.
Например, были предложены эволюционные алгоритмы~\cite{DBLP:journals/pami/LucasR05,DBLP:conf/cec/Gomez06}, муравьиные алгоритмы~\cite{chivilikhin-12-ant}.
Метаэвристические алгоритмы также являются неточными~--- ими вообще не гарантируется, что какое-то решение будет найдено за конечное время.

Голландские ученые Хойл и Вервер в 2010 году предложили алгоритм \texttt{DFASAT}, способный гарантированно генерировать ДКА минимального размера по произвольным данным~\cite{heule-icgi10}.
Алгоритм основан на сведении задачи генерации ДКА по примерам поведения к задаче выполнимости булевой формулы (SAT), где в качестве промежуточного шага используется сведение к задаче раскраски графа, которое было предложено еще в 1997 году~\cite{Coste97regularinference}.
Для сокращения пространства поиска при решении задачи выполнимости Хойл и Вервер предложили ряд подходов к нарушению симметрии: граф совместимости, дополнительные дизъюнкты, поиск большой клики.
Однако, даже с использованием всех этих техник, \texttt{DFASAT} способен за разумное время строить автоматы с не более чем десятью состояниями.
Хойл и Вервер для упрощения задачи предложили делать предварительно несколько слияний в префиксном дереве с помощью алгоритма \texttt{EDSM}, однако тогда теряется точность~--- гарантия минимальности найденного ДКА.

Позднее автором диссертации совместно с научным руководителем Ульянцевым В.~И. и профессором Шалыто А.~А. были предложены предикаты нарушения симметрии на основе алгоритма обхода графа в ширину (BFS)~\cite{zakirzyanov2015LATA}.
Для каждого ДКА с $M$ состояниями существует $\mathcal{O}\left(M!\right)$ изоморфных ему автоматов, которые, с точки зрения программного средства для решения SAT, являются различными.
Предложенные предикаты нарушения симметрии позволяют значительно сократить пространство поиска при решении SAT~--- вместо факториала изоморфных автоматов при решении перебирается только единственный представитель класса эквивалентности по изоморфизму~--- ДКА, пронумерованный в порядке BFS-обхода.
Данный метод является наиболее эффективным известным точным методом генерации ДКА по заданным примерам поведения и позволяет строить автоматы с числом состояний до сорока.
Однако, анализ показал, что предложенное наивное кодирование BFS-предикатов на языке SAT неоптимально~--- получаемая булева формула слишком велика.
Помимо этого, не рассматривались предикаты нарушения симметрии, основанные на кодировании алгоритма обхода графа в глубину (DFS).

Среди прочих недостатков методов генерации ДКА, основанных на сведении к SAT, можно отметить зависимость размера булевой формулы от числа имеющихся примеров поведения.
Зачастую множества примеров поведения избыточны и содержат лишние слова.
Анализ литературы показал, что интеллектуальных методов выбора подмножества примеров поведения ранее не предлагалось.
Возможным решением может быть применение подхода \emph{уточнения абстракции по контрпримерам} (\emph{counterexample-guided abstraction refinement}~--- CEGAR)~--- итеративного алгоритма, изначально разработанного для построения модели программного обеспечения~\cite{DBLP:conf/cav/ClarkeGJLV00,мандрыкин2013введение}.

Обратной ситуацией является генерация ДКА по небольшим множествам примеров поведения.
В таком случае полезным может быть генерация всех соответствующих неизоморфных ДКА с минимальным числом состояний для проведения дальнейшего анализа.
Также, существование единственного автомата минимального размера говорит о качестве имеющихся обучающих данных.
Однако, задача генерации всех неизоморфных ДКА ранее не ставилась, и, соответственно, эффективных методов ее решения предложено не было.

\textbf{Целью} настоящей диссертации является повышение эффективности точных методов генерации детерминированных конечных автоматов по заданным примерам поведения посредством сокращения пространства поиска при решении задачи выполнимости.

Для достижения указанной цели в работе поставлены и решены следующие \textbf{задачи}:
\begin{enumerate}
  \item Разработка предикатов нарушения симметрии, основанных на кодировании алгоритмов обхода графа в ширину и в глубину, для сокращения пространства поиска при решении задачи выполнимости.
  Разработка и реализация точных методов генерации ДКА по заданным примерам поведения, использующих данные предикаты, проведение экспериментальных исследований разработанных методов.

  \item Разработка и реализация точного метода генерации ДКА по избыточному набору примеров поведения с использованием сведения к задаче выполнимости и подхода уточнения абстракции по контрпримерам.
  Проведение экспериментальных исследований разработанного метода.
  
  \item Разработка и реализация метода генерации всех неизоморфных ДКА минимального размера, удовлетворяющих заданным примерам поведения, с использованием предикатов нарушения симметрии и программных средств решения задачи выполнимости.
  Проведение экспериментальных исследований разработанного метода.
\end{enumerate}

\textbf{Предмет исследования}~{---} методы точной генерации ДКА, использующие программные средства для решения задачи выполнимости.

\textbf{Соответствие паспорту специальности.} Данная диссертация соответствует пункту 10 <<Разработка основ математической теории языков и грамматик, теории конечных автоматов и теории графов>> паспорта специальности 05.13.17~--- «Теоретические основы информатики».

\textbf{Основные положения, выносимые на~защиту:}
\begin{enumerate}
  \item \emph{Подход} к построению предикатов нарушения симметрии, основанных на кодировании алгоритмов обхода графа в ширину и в глубину, для сокращения пространства поиска при решении задачи выполнимости.
  Точные \emph{методы} генерации ДКА по заданным примерам поведения, использующие данные предикаты.
  
  \item Точный \emph{метод} генерации ДКА по избыточному набору примеров поведения с использованием сведения к задаче выполнимости и подхода уточнения абстракции по контрпримерам.

  \item \emph{Метод} генерации всех неизоморфных ДКА минимального размера, удовлетворяющих заданным примерам поведения, с использованием предикатов нарушения симметрии и программных средств решения задачи выполнимости.
\end{enumerate}

\textbf{Научная новизна} диссертации состоит в~следующем:
\begin{enumerate}
  \item Предикаты нарушения симметрии, основанные на кодировании алгоритма обхода графа в глубину ранее не предлагались.
  Предложенные предикаты нарушения симметрии, основанные на алгоритме обхода графа в ширину, выражаются булевой формулой, состоящей из асимптотически меньшего числа дизъюнктов сравнительно с существующим подходом.
  Помимо этого, впервые предложены предикаты нарушения симметрии, использующие особенности дерева обхода графа в ширину.

  \item Точных методов генерации ДКА применимых в случае, когда число примеров поведения излишне велико, ранее не предлагалось.
  Использование подхода уточнения абстракции по контрпримерам одновременно со сведением к задаче выполнимости позволяет генерировать ДКА по избыточному набору примеров поведения путем итеративного добавления только значимых примеров до тех пор, пока не будет получен ДКА, соответствующий всему избыточному набору.

  \item Методов для генерации всех неизоморфных ДКА минимального (или любого фиксированного) размера, удовлетворяющих заданным примерам поведения, ранее не предлагалось.

\end{enumerate}

\textbf{Методология и методы исследования.} 
Методологическую основу диссертации составили принципы формализации, обобщения, дедуктивного и индуктивного обоснования утверждений, проведение экспериментальных исследований и анализ их результатов.
В работе используются методы теории автоматов, теории графов, теории сложности, дискретной математики, объектно-ориентированное программирование, методы проведения и анализа экспериментальных исследований.

\textbf{Достоверность} полученных результатов подтверждается корректным обоснованием постановок задач, точной формулировкой критериев, а также результатами проведенных экспериментальных исследований по использованию предложенных в диссертации методов.

\textbf{Теоретическая значимость работы} заключается в том, что в ней для точных методов генерации ДКА предложены новые подходы к сокращению пространства поиска при решении задачи выполнимости~--- предикаты нарушения симметрии на основе кодирования алгоритмов обхода графа в глубину и в ширину:
\begin{itemize}
  \item предложен способ кодирования свойства DFS-пронумерованности ДКА на языке SAT;
  \item предложен новый способ кодирования свойства BFS-пронумерованности ДКА на языке SAT, требующий асимптотически меньшего числа дизъюнктов, относительно известных способов;
  \item предложен способ кодирования различных свойств BFS-дерева на языке SAT.
\end{itemize}
Помимо этого, предложен способ объединить метод генерации ДКА при помощи сведения к задаче выполнимости с подходом уточнения абстракции по контрпримерам.
Также, разработанные предикаты нарушения симметрии позволили разработать метод для решения задачи генерации всех неизоморфных ДКА минимального размера, которая ранее не имела эффективных методов решения.

\textbf{Практическая значимость работы} состоит в повышении эффективности точных методов генерации ДКА по заданным примерам поведения. 
Экспериментально показано, что разработанный метод генерации ДКА по заданным примерам поведения с использованием предикатов нарушения симметрии на основе алгоритма BFS является самым эффективным по времени генерации автомата среди известных на настоящий момент точных методов и позволяет генерировать автоматы большего размера относительно методов, предложенных ранее.
Разработанный точный метод генерации ДКА по избыточному набору примеров поведения с использованием сведения к SAT и подхода CEGAR позволяет эффективно генерировать автоматы в случае, когда набор примеров поведения слишком объемен и получаемая булева формула слишком велика для современных программных средств для решения SAT.
Разработанный метод генерации всех неизоморфных ДКА минимального размера, удовлетворяющих заданным примерам поведения, с использованием предикатов нарушения симметрии и программных средств для решения SAT, является первым известным методом для генерации всех неизоморфных автоматов.
Также с его помощью можно оценить полноту имеющихся данных, выраженных в виде примеров поведения, путем доказательства или опровержения существования единственного ДКА минимального размера, описывающего эти данные.

Помимо этого, все разработанные методы и подходы в дальнейшем могут быть адаптированы для задач генерации более сложных конечно-автоматных моделей~\cite{ulyantsev-phd-13}.
Так, например, предложенный метод генерации всех неизоморфных ДКА был адаптирован для генерации конечных автоматов, моделирующих поведение программируемых логических контроллеров~\cite{DBLP:journals/tii/ChivilikhinPCCV20}.

\textbf{Внедрение результатов работы.}
Результаты работы использовались при выполнении проекта SAUNA (``Integrated safety assessment and justification of nuclear power plant automaton''), выполненного исследовательской группой ``IT in Industrial Automation'' кафедры электротехники и автоматики университета Аалто, Финляндия, в рамках Финской программы исследований безопасности атомных электростанций~--- SAFIR2018\footnote{http://safir2018.vtt.fi/}.
В частности, одной из задач проекта была разработка метода генерации моделей различных компонентов системы управления атомных электростанций по заданным примерам поведения и спецификации выраженной на языке темпоральной логики.
Данная задача была решена с использованием подхода уточнения абстракции по контрпримерам способом, аналогичным предложенному автором диссертации для генерации ДКА, что подтверждается письмом руководителя исследовательской группы ``IT in Automation'' В.~В. Вяткина.

Результаты работы также использовались при выполнении под руководством автора диссертации гранта Российского фонда фундаментальных исследований (проект 18-37-00425 <<Разработка эффективных методов машинного обучения для построения детерминированных конечных автоматов на основе решения задачи выполнимости>>, 2018--2020 гг.).

Полученные результаты также использовались в рамках государственной финансовой поддержки ведущих университетов Российской Федерации, субсидия 074-U01 (НИР <<Биоинформатика, машинное обучение, технологии программирования, теория кодирования, проактивные системы>>, 2013--2017 гг.) и субсидия 08-08 (НИР <<Методы, модели и технологии искусственного интеллекта в биоинформатике, социальных медиа, киберфизических, биометрических и речевых системах>>, 2018--2020 гг.)

Результаты работы также внедрены в учебный процесс факультета информационных технологий и программирования Университета ИТМО в рамках курса <<Проектирование автоматных программ>> программы бакалавриата <<Математические модели и алгоритмы в разработке программного обеспечения>>, что подтверждается актом об использовании.

\textbf{Апробация результатов работы.}
Основные результаты работы докладывались на следующих конференциях и семинарах:
\begin{enumerate}
  \item 9\textsuperscript{th} International Conference on Language and Automata Theory and Applications (LATA 2015). 2015, Ницца, Франция.
  \item 6\textsuperscript{th} International Symposium ``From Data to Models and Back (DataMod)''. 2017, Тренто, Италия.
  \item 16\textsuperscript{th} IEEE International Conference on Industrial Informatics (INDIN 2018). 2018, Порту, Португалия.
  \item 13\textsuperscript{th} International Conference on Language and Automata Theory and Applications (LATA 2019). 2019, Санкт-Петербург.
  \item IV-VII Всероссийский конгресс молодых ученых. 2015-2018, Санкт-Петербург.
  \item IX Конгресс молодых ученых. 2020, Санкт-Петербург.
  \item XLVI Научная и учебно-методическая конференция Университета \mbox{ИТМО}. 2017, Санкт-Петербург.
  \item XLVIII Научная и учебно-методическая конференция Университета ИТМО. 2019, Санкт-Петербург.
\end{enumerate}

\textbf{Личный вклад автора.}
Идея предикатов нарушения симметрии на основе кодирования алгоритма обхода графа в глубину, идея метода генерации ДКА по заданным примерам поведения, использующего их, а также реализация алгоритма на базе предложенного метода и проведение вычислительных экспериментов принадлежит лично автору.
Идея предикатов нарушения симметрии на основе алгоритма обхода графа в ширину, кодирование которых требует асимптотически меньшего числа дизъюнктов, идея кодирования свойств BFS-дерева на языке SAT и идея метода генерации ДКА, использующего данные разработки, принадлежат совместно автору диссертации и Ж.~Маркешу-Сильве; реализация алгоритмов на базе предложенных методов принадлежит лично автору, проведение вычислительных экспериментов принадлежит совместно автору диссертации и А.И.~Игнатьеву.
Идея точного метода генерации ДКА по избыточному набору примеров поведения с использованием сведения к задаче выполнимости и подхода уточнения абстракции по контрпримерам, реализация алгоритма на базе предложенного метода и проведение вычислительных экспериментов принадлежит лично автору.
Идея точного метода генерации всех ДКА минимального размера по заданным примерам поведения с использованием программных средств решения задачи выполнимости принадлежит совместно автору диссертации и научному руководителю В.И.~Ульянцеву, реализация алгоритма на базе предложенного метода и проведение вычислительных экспериментов принадлежит лично автору.
В работах, выполненных в соавторстве, В.И~Ульянцевым осуществлялось общее руководство исследованиями.

\textbf{Публикации.}
Основные результаты по теме диссертации изложены в десяти публикациях, из них четыре опубликованы в изданиях, индексируемых в базе цитирования Scopus, одна публикация издана в журнале, рекомендованном ВАК.
Также у автора диссертации имеется одна публикация по другой теме из области машинного обучения, опубликованная в издании, индексируемом в базе цитирования Scopus.