%!TEX root = ../dissertation.tex
\usepackage{etoolbox}
\usepackage{pdflscape}
\usepackage{geometry}
\usepackage{xparse}

%% Various maths
\usepackage{amsthm,amsmath,amscd,amsfonts,amssymb,mathtools}

%% Localization
\usepackage[main=russian,german,english]{babel}
\usepackage{fontspec}
\usepackage{iflang}


%% Fonts
\setmonofont{Courier New}
\defaultfontfeatures{Ligatures=TeX}
\setmainfont{Times New Roman}
\setsansfont{Arial}

%% Page geometry
\geometry{a5paper, top=14mm, bottom=14mm, inner=18mm, outer=10mm, footskip=5mm, nomarginpar}
\setlength{\topskip}{0pt}
\setlength{\footskip}{12.3pt}
\SingleSpacing

\makeevenhead{plain}{}{\thepage}{}
\makeoddhead{plain}{}{\thepage}{}
\makeevenfoot{plain}{}{}{}
\makeoddfoot{plain}{}{}{}
\pagestyle{plain}

%% Penalties
\tolerance=1414
\hbadness=1414
\emergencystretch=1.5em
\hfuzz=0.3pt
\vfuzz=\hfuzz
\clubpenalty=10000
\widowpenalty=10000
\brokenpenalty=4991

%% Tuning of Table of Contents
\renewcommand{\cftchapterdotsep}{\cftdotsep}
\setrmarg{2.55em plus1fil}
\renewcommand{\cftchapterpagefont}{\normalfont}
\renewcommand{\cftchapterleader}{\cftdotfill{\cftchapterdotsep}}
\renewcommand{\cftchapteraftersnum}{.\space}
\AtBeginDocument{\setsecnumformat{\csname the#1\endcsname\quad}}
\renewcommand*{\cftappendixname}{\appendixname\space}

%% Tuning of List of Figures/Tables
\makeatletter
\renewcommand{\@tocrmarg}{4em}
\renewcommand{\@pnumwidth}{3em}
\makeatother

%% Fonts and intervals of the basic things
\newcommand{\basegostsectionfont}{\fontsize{10pt}{12pt}\selectfont\bfseries}
\newlength{\gostindent}
\setlength{\gostindent}{30pt}
\setbeforesecskip{\gostindent}
\setaftersecskip{\gostindent}
\setbeforesubsecskip{\gostindent}
\setaftersubsecskip{\gostindent}
\setbeforesubsubsecskip{\gostindent}
\setaftersubsubsecskip{\gostindent}

\makechapterstyle{thesisgost}{%
\chapterstyle{default}%
\setlength{\beforechapskip}{0pt}%
\setlength{\midchapskip}{0pt}%
\setlength{\afterchapskip}{\gostindent}%
\renewcommand*{\chapnamefont}{\basegostsectionfont}%
\renewcommand*{\chapnumfont}{\basegostsectionfont}%
\renewcommand*{\chaptitlefont}{\basegostsectionfont}%
\renewcommand*{\chapterheadstart}{}%
\renewcommand*{\afterchapternum}{\quad}%
\renewcommand*{\printchapternum}{\centering\chapnumfont\thechapter}%
\renewcommand*{\printchaptername}{}%
\renewcommand*{\printchapternonum}{\centering}}

\makeatletter
\makechapterstyle{thesisgostchapname}{%
    \chapterstyle{thesisgost}
    \renewcommand*{\printchapternum}{\chapnumfont\thechapter}
    \renewcommand*{\printchaptername}{\centering\chapnamefont\@chapapp} %
}
\makeatother

\chapterstyle{thesisgost}
\setsecheadstyle{\basegostsectionfont\centering}
\setsecindent{0pt}
\setsubsecheadstyle{\basegostsectionfont\centering}
\setsubsecindent{0pt}
\setsubsubsecheadstyle{\basegostsectionfont\centering}
\setsubsubsecindent{0pt}
\sethangfrom{\noindent #1}

\chapterstyle{thesisgostchapname}
\renewcommand*{\cftchaptername}{\chaptername\space}

%% Making all the counters global
\counterwithout{equation}{chapter}
\counterwithout{equation}{section}
\counterwithout{equation}{subsection}
\counterwithout{figure}{chapter}
\counterwithout{figure}{section}
\counterwithout{figure}{subsection}
\counterwithout{table}{chapter}
\counterwithout{table}{section}
\counterwithout{table}{subsection}

\AtBeginDocument{%
\regtotcounter{totalcount@figure}%
\regtotcounter{totalcount@table}%
\regtotcounter{TotPages}%
}

%% Some not yet used magic to form Russian messages about sizes and counts
%% http://www.linux.org.ru/forum/general/6993203#comment-6994589 (используется totcount)
\makeatletter
\def\formbytotal#1#2#3#4#5{%
    \newcount\@c
    \@c\totvalue{#1}\relax
    \newcount\@last
    \newcount\@pnul
    \@last\@c\relax
    \divide\@last 10
    \@pnul\@last\relax
    \divide\@pnul 10
    \multiply\@pnul-10
    \advance\@pnul\@last
    \multiply\@last-10
    \advance\@last\@c
    \total{#1}~#2%
    \ifnum\@pnul=1#5\else%
    \ifcase\@last#5\or#3\or#4\or#4\or#4\else#5\fi
    \fi
}
\makeatother

%% A special environment for locale-dependent commands
%% Usage: \newlocalizedcommand{\YourCommandName}{expansion in English}{expansion in Russian}
%%        \renewlocalizedcommand{\YourCommandName}{expansion in English}{expansion in Russian}
\newcommand{\newlocalizedcommand}[3]{\newcommand{#1}{\IfLanguageName{russian}{#3}{#2}}}
\newcommand{\renewlocalizedcommand}[3]{\renewcommand{#1}{\IfLanguageName{russian}{#3}{#2}}}

%% Theorems (localized) %%
\newlocalizedcommand{\definitionname}{Definition}{Определение}
\newlocalizedcommand{\corollaryname}{Corollary}{Следствие}
\newlocalizedcommand{\theoremname}{Theorem}{Утверждение}
\newlocalizedcommand{\lemmaname}{Lemma}{Лемма}

\theoremstyle{definition}
\newtheorem{definition}{\definitionname}
\newtheorem{theorem}{\theoremname}
\newtheorem{lemma}{\lemmaname}
\newtheorem{corollary}{\corollaryname}

%% Paragraph formatting %%
\usepackage{indentfirst}
\AtBeginDocument{\setlength{\parindent}{2.5em}}

%% Enumerations (partially localized) %%
\usepackage{enumitem}
\setlist{nosep,labelindent=\parindent,leftmargin=*}

\makeatletter
\def\asbukx#1{\expandafter\@asbukx\csname c@#1\endcsname}
\def\@asbukx#1{\ifcase#1\or a\or б\or в\or г\or д\or е\or ж\or и\or к\or л\or м\or н\or п\or р\or с\or т\or у\or ф\or х\or ц\or ш\or щ\or э\or ю\or я\fi}
\def\Asbukx#1{\expandafter\@Asbukx\csname c@#1\endcsname}
\def\@Asbukx#1{\ifcase#1\or А\or Б\or В\or Г\or Д\or Е\or Ж\or И\or К\or Л\or М\or Н\or П\or Р\or С\or Т\or У\or Ф\or Х\or Ц\or Ш\or Щ\or Э\or Ю\or Я\fi}
\AddEnumerateCounter{\Asbukx}{\@Asbukx}{М}
\AddEnumerateCounter{\asbukx}{\@asbukx}{м}
\makeatother

\renewcommand{\labelitemi}{\normalfont\bfseries{--}}
\renewcommand\labelenumii{\arabic{enumii})}
\renewcommand\theenumii{\arabic{enumii}}
\renewlocalizedcommand{\labelenumi}{\alph{enumi})}{\asbukx{enumi})}
\renewlocalizedcommand{\theenumi}{\alph{enumi}}{\asbukx{enumi}}


%% Tuning of how floats look like
% We use floatrow/caption instead of memoir's poorly-working built-ins
\let\newfloat\undefined
\usepackage{caption}
\usepackage{floatrow}
\usepackage{subcaption}

%% Babel uses its own way to control captions, adhere to it
\addto{\captionsenglish}{%
\renewcommand{\figurename}{Figure}%
\renewcommand{\contentsname}{Contents}%
%\renewcommand{\ALG@name}{Algorithm}
}
\addto{\captionsrussian}{%
\renewcommand{\figurename}{Рисунок}%
\renewcommand{\contentsname}{Содержание}%
%\makeatletter
%\renewcommand{\ALG@name}{Листинг}%
%\makeatother
\renewcommand{\algorithmname}{Листинг}%
}

\floatsetup[figure]{style=plain, capposition=bottom}
\captionsetup[figure]{
    labelsep=endash,
    singlelinecheck=false,
    labelfont={normalsize,md},
    justification=centering,
    position=bottom
}
\floatsetup[table]{style=plain, capposition=top}
\captionsetup[table]{
    labelsep=endash,
    singlelinecheck=false,
    labelfont={normalsize,md},
    justification=justified,
    position=top
}
\floatsetup[algorithm]{style=plain, capposition=top}
\captionsetup[algorithm]{
    labelsep=endash,
    singlelinecheck=false,
    labelfont={normalsize,md},
    justification=justified,
    position=top
}
%\floatsetup[lstlisting]{style=plain, capposition=top}
%\captionsetup[lstlisting]{
%    labelsep=endash,
%    singlelinecheck=false,
%    labelfont={normalsize,md},
%    justification=justified,
%    position=top
%}

%% Tuning of table-of-contents

\settocdepth{subsection}            % до какого уровня подразделов выносить в оглавление
\setsecnumdepth{subsection}         % до какого уровня нумеровать подразделы

%% Custom math fonts

\usepackage{mathrsfs}

%% Graphics

\usepackage[dvipsnames, table, hyperref, cmyk]{xcolor}
\usepackage{graphicx}
\usepackage{pgfplots}
\pgfplotsset{compat=newest} 

% Include articles
\usepackage[final]{pdfpages}

%% Tables %%
\usepackage{tabularx}
\usepackage{longtable}
\usepackage{multirow,makecell}
\usepackage{hhline}
\usepackage{adjustbox}

\newcommand{\specialcell}[2][c]{%
  \begin{tabular}[#1]{@{}c@{}}#2\end{tabular}}

%% Hyperref %%
\usepackage{hyperref}
\definecolor{linkcolor}{rgb}{0,0,0}
\definecolor{citecolor}{rgb}{0,0,0}
\definecolor{urlcolor}{rgb}{0,0,0}

% No hypersetup here, as it needs some information not available here

%% Algorithmic environments %%
%\usepackage[linesnumbered,lined,boxed,commentsnumbered]{algorithm2e}
\usepackage{algorithm}
%\usepackage{algorithmic}
\usepackage[noend]{algpseudocode}
%\usepackage{amsmath}, amsthm,amsfonts,amssymb,amscd}

\algrenewcommand\algorithmicrequire{\textbf{Input:}}
\algrenewcommand\algorithmicensure{\textbf{Output:}}
\algnewcommand\algorithmicto{\textbf{to}}
\algrenewtext{For}[3] {\algorithmicfor\ $#1 \gets #2$ \algorithmicto\ $#3$ \algorithmicdo}
\algnewcommand\Continue{\textbf{continue}}
\algnewcommand\AndL{\textbf{and} }
\algnewcommand\OrL{\textbf{or} }
\algnewcommand\True{\textbf{True}}
\algnewcommand\False{\textbf{False}}

%% Counters
\usepackage[figure,table]{totalcount}
\usepackage{totcount}
\usepackage{totpages}

%% Misc localization commands %%
\let\origle\le   \renewlocalizedcommand{\le}{\origle}{\leqslant}
\let\origleq\leq \renewlocalizedcommand{\leq}{\origleq}{\leqslant}
\let\origge\ge   \renewlocalizedcommand{\ge}{\origge}{\geqslant}
\let\origgeq\geq \renewlocalizedcommand{\geq}{\origgeq}{\geqslant}
\let\origtan\tan \renewlocalizedcommand{\tan}{\origtan}{\operatorname{tg}}
\let\origcot\cot \renewlocalizedcommand{\cot}{\origcot}{\operatorname{ctg}}
\let\origcsc\csc \renewlocalizedcommand{\csc}{\origcsc}{\operatorname{cosec}}
\let\origempty\emptyset \renewlocalizedcommand{\emptyset}{\origempty}{\varnothing}

%% Misc technical things %%
\newcommand{\resetfloatcounters}{%
\setcounter{figure}{0}%
\setcounter{table}{0}%
\setcounter{theorem}{0}%
\setcounter{lemma}{0}%
\setcounter{definition}{0}%
\setcounter{corollary}{0}%
\setcounter{footnote}{0}%
}

%% Bibliography packages and configuration

\usepackage{csquotes} % biblatex рекомендует его подключать. Пакет для оформления сложных блоков цитирования.

%%% Загрузка пакета с основными настройками %%%
\makeatletter
\usepackage[backend=biber, bibencoding=utf8, sorting=none, style=gost-numeric, language=autobib,
            autolang=other, defernumbers=true, sortcites=true, doi=false, isbn=false, movenames=false, maxnames=99]{biblatex}
\ltx@iffilelater{biblatex-gost.def}{2017/05/03}%
{%\toggletrue{bbx:gostbibliography}%
\renewcommand*{\revsdnamepunct}{\addcomma}}{}
\makeatother

\renewcommand*{\newblockpunct}{\addperiod\addnbspace---\space\bibsentence}

\DefineBibliographyStrings{english}{pages = {p\adddot}}

\DefineBibliographyExtras{russian}{%
  \protected\def\bibrangedash{--\penalty\value{abbrvpenalty}}% almost unbreakable dash
  \protected\def\bibdaterangesep{\bibrangedash}%тире для дат
}
\DefineBibliographyExtras{english}{%
  \protected\def\bibrangedash{--\penalty\value{abbrvpenalty}}% almost unbreakable dash
  \protected\def\bibdaterangesep{\bibrangedash}%тире для дат
}

%Set higher penalty for breaking in number, dates and pages ranges
\setcounter{abbrvpenalty}{10000} % default is \hyphenpenalty which is 12
%Set higher penalty for breaking in names
\setcounter{highnamepenalty}{10000} % If you prefer the traditional BibTeX behavior (no linebreaks at highnamepenalty breakpoints), set it to ‘infinite’ (10 000 or higher).
\setcounter{lownamepenalty}{10000}

%% An environment which rewrites \cite to be \footfullcite for citations not in the author's list
\makeatletter
\newtoggle{footnotized@value}\togglefalse{footnotized@value}

\DeclareDocumentCommand{\trickycite}{oom}{%
\filteredcite{#3}%
\IfNoValueTF{#2}{%
\IfNoValueTF{#1}{%
% no #1 no #2
\iftoggle{footnotized@value}{\unspace\footfullcite{#3}}{\oldcite{#3}}%
}{%
% yes #1 no #2
\iftoggle{footnotized@value}{\unspace\footfullcite[#1]{#3}}{\oldcite[#1]{#3}}%
}}{
% yes #1 yes #2
\iftoggle{footnotized@value}{\unspace\footfullcite[#1][#2]{#3}}{\oldcite[#1][#2]{#3}}%
}}%
\newenvironment{footnotizeexcept}[1]{\begingroup%
\DeclareCiteCommand{\filteredcite}{}{\ifkeyword{#1}{\global\togglefalse{footnotized@value}}{\global\toggletrue{footnotized@value}}}{}{}%
\let\oldcite\cite\let\cite\trickycite}{\let\cite\oldcite\endgroup}
\makeatother

\DeclarePairedDelimiter\abs{\lvert}{\rvert}
\definecolor{darkred}{rgb}{0.7,0.1,0.1}
\definecolor{middarkgrey}{rgb}{0.35,0.35,0.35}

\newcommand{\inote}[1]{\medskip\noindent$[$\textcolor{darkred}{Илья}: \emph{\textcolor{middarkgrey}{#1}}$]$\medskip}
\newcommand{\pto}{\mathrel{\ooalign{\hfil$\mapstochar$\hfil\cr$\to$\cr}}}